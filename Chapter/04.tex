\documentclass[../Marcus.tex]{subfiles}

\begin{document}

\chapter{Galois Theory Applied to Prime Decomposition}

\subsection*{Exercise 4.1}

Let $\sigma\in D(Q/P)$ and $\tau\in E(Q/P)$ be given. Then for each $\alpha\in S$, we have $\tau\sigma^{-1}(\alpha)\equiv \sigma^{-1}(\alpha) \pmod{Q}$, i.e., $\tau\sigma^{-1}(\alpha)-\sigma^{-1}(\alpha)\in Q$. So $\sigma(\tau\sigma^{-1}(\alpha)-\sigma^{-1}(\alpha))=\sigma\tau\sigma^{-1}(\alpha)-\alpha \in Q$. This means $\sigma\tau\sigma^{-1}(\alpha) \equiv \alpha \pmod{Q}$ and hence $\sigma\tau\sigma^{-1}\in E(Q/P)$.

\subsection*{Exercise 4.2}

Note that $PS=PS_DS_ES=(P'_1\cdots P'_r)S_ES=(P''_1\cdots P''_r)S$. And since $PS=(Q_1\cdots Q_r)^e$, so by the uniqueness of prime decompositions, we have $P''_iS=Q_i^e$ for all $i=1,\ldots,r$. (After rearranging appropriately, of course.)

\subsection*{Exercise 4.3}

(a) Let $g$ be a generator of $(\ZZ/p\ZZ)^\times$. Consider the equation $x^2\equiv -1\pmod{p}$. Write $x=g^k,1\leq k\leq p-1$, and $-1=g^{(p-1)/2}$, then $g^{2k}\equiv g^{(p-1)/2}\pmod{p}$. If it has solutions, then we must have $(p-1)/2\mid 2k$. So the only possible solutions to $k$ are $(p-1)/4$ and $3(p-1)/4$. And these two are integers if and only if $p\equiv 1\pmod{4}$. This completes the proof.

Note that since $(\ZZ/p\ZZ)^\times=\{g,g^2,g^3,\ldots,g^{p-1}\}$, so the squares in $(\ZZ/p\ZZ)^\times$ are precisely $\{g^2,g^4,g^6,\ldots,g^{p-1}\}$. This means for $g^n\in(\ZZ/p\ZZ)^\times$, $(g^n/p)=1$ if and only if $n$ is even. So if we write $a=g^s,b=g^t$ and $ab=g^{s+t}$. Then $(ab/p)=1 \iff s+t$ is even $\iff s,t$ have the same parity $\iff (a/p)=(b/p)$. Hence, $(a/p)(b/p)=(ab/p)$.

(b) First, consider the case where $q=2$. This is easily seen by using Theorem 25 (p. 52). So $(2/p)=1 \iff$ $2$ splits completely in $\QQ(\sqrt{\pm p}) \iff \pm p\equiv 1 \pmod{8} \iff p\equiv \pm1 \pmod{8}$.

Now, assume $q\neq p$ is an odd prime. We split it into three cases and use Theorem 25 again.

Case 1: $p\equiv 1\pmod{4}$. Then $(q/p)=1 \iff$ $q$ splits completely in $\QQ(\sqrt{p}) \iff$ $p$ is a square mod $q$ $\iff (p/q)=1$. So we have $(q/p)=(p/q)$.

Case 2: $p\equiv 3\pmod{4}$ and $q\equiv 1\pmod{4}$. Then $(q/p)=1 \iff$ $q$ splits completely in $\QQ(\sqrt{-p}) \iff$ $-p$ is a square mod $q$ $\iff (-p/q)=1 \iff (-1/q)(p/q)=(p/q)=1$. So we have $(q/p)=(p/q)$. (We've used (a) in the last step.)

Case 2: $p\equiv q \equiv3\pmod{4}$. Then $(q/p)=1 \iff$ $q$ splits completely in $\QQ(\sqrt{-p}) \iff$ $-p$ is a square mod $q$ $\iff (-p/q)=1 \iff (-1/q)(p/q)=-(p/q)=1$. So we have $(q/p)=-(p/q)$. (We've again used (a) in the last step.)

\subsection*{Exercise 4.4}

For (a), (b), (c) and (d), write $b=\prod p_i^{r_i}$ and $b'=\prod q_j^{s_j}$.

(a) From Exercise 4.3(a), we know $(a/p)(b/p)=(ab/p)$ when $p$ is an odd prime. For the first one,
\begin{align*}
    \left(\dfrac{aa'}{b}\right) &= \left(\dfrac{aa'}{\prod p_i^{r_i}}\right) \overset{\df}{=} \prod\left(\dfrac{aa'}{p_i}\right)^{r_i} \\
    &= \prod \left(\dfrac{a}{p_i}\right)^{r_i}\left(\dfrac{a'}{p_i}\right)^{r_i}
    \overset{\df}{=} \left(\dfrac{a}{\prod p_i^{r_i}}\right)\left(\dfrac{a'}{\prod p_i^{r_i}}\right) = \left(\dfrac{a}{b}\right)\left(\dfrac{a'}{b}\right)
\end{align*}
And for the second one,
\begin{align*}
    \left(\dfrac{a}{bb'}\right) &= \left(\dfrac{a}{\prod p_i^{r_i}\prod q_j^{s_j}}\right) \overset{\df}{=} \prod\left(\dfrac{a}{p_i}\right)^{r_i}\prod\left(\dfrac{a}{q_j}\right)^{s_j} \\
    &\overset{\df}{=} \left(\dfrac{a}{\prod p_i^{r_i}}\right)\left(\dfrac{a}{\prod q_j^{s_j}}\right) = \left(\dfrac{a}{b}\right)\left(\dfrac{a}{b'}\right)
\end{align*}

(b) We know $(a/p)=(a'/p)$ for Legendre symbols if $a\equiv a'\pmod{p}$. Note that if $a\equiv a'\pmod{b}$, then $a\equiv a'\pmod{p_i}$ for all $i$. So $(a/b)\overset{\df}{=}\prod (a/p_i)^{r_i}=\prod (a'/p_i)^{r_i}\overset{\df}{=}(a'/b)$.

(c) Let $p_1,\ldots,p_n$ be all prime divisors of $b$ of the form $4k+3$. This means the Legendre symbol $(-1/p_i)=-1$ for all $i=1,\ldots,n$. The result now follows from the observation that $(-1/b)\overset{\df}{=}\prod (-1/p_i)^{r_i}=1 \iff \#\{r_1,\ldots,r_n \text{ is odd}\}$ is even $\iff b=\prod p_i^{r_i}\equiv 1\pmod{4}$.

(d) Let $p_1,\ldots,p_n$ be all prime divisors of $b$ of the form $8k\pm3$. This means the Legendre symbol $(2/p_i)=-1$ for all $i=1,\ldots,n$. The result now follows from the observation that $(2/b)\overset{\df}{=}\prod (2/p_i)^{r_i}=1 \iff \#\{r_1,\ldots,r_n \text{ is odd}\}$ is even $\iff b=\prod p_i^{r_i}\equiv \pm1\pmod{8}$.

(e) Write $a=\prod p_i$ where $p_i$ not necessarily distinct and $b=\prod q_j$ also not necessarily distinct. Observe that $(a/b)(b/a)\overset{(a)}{=}\prod (p_i/q_j)(q_j/p_i)=(-1)^k$ where $k$ is the number of pairs where $p_i$ and $q_j\equiv 3\pmod{4}$. Moreover, $(-1)^k=(-1)^{mn}$ where $m$ and $n$ are the numbers of $p_i$ and $q_j\equiv 3\pmod{4}$, respectively. Note that $(-1)^{mn}=1 \iff$ at least one of $m,n$ is even $\iff$ at least one of $a,b\equiv 1\pmod{4}$. This shows that $(a/b)(b/a)=1 \iff$ at least one of $a,b\equiv 1\pmod{4}$, which completes the proof.

(f) We don't need to bother factoring $2413$, just simply apply the results we have shown. (In fact, $2413 = 19\times 127$.)
\begin{align*}
    \left(\dfrac{2413}{4903}\right) &\overset{(e)}{=} \left(\dfrac{4903}{2413}\right) \overset{(b)}{=} \left(\dfrac{77}{2413}\right) \overset{(a)}{=} \left(\dfrac{7}{2413}\right)\left(\dfrac{11}{2413}\right) \\ &\overset{(e)}{=} \left(\dfrac{2413}{7}\right)\left(\dfrac{2413}{11}\right) \overset{(b)}{=} \left(\dfrac{5}{7}\right)\left(\dfrac{4}{11}\right) = -1
\end{align*}

\subsection*{Exercise 4.5}

We use the same notation as in the textbook. (See Theorem 28 (p. 70) and Theorem 29 (p. 73).) These two are also the main results we are going to use. In addition, for the followings, we fix a prime $Q$ in $L$.

(a) Suppose $P$ is inert in $L$. This means $r=e=1$ and so $[L_D:K]=[L:L_E]=1$. By Galois theory, we have $[G:D]=[E:\{\id_L\}]=1$, i.e., $G=D$ and $E=\{\id_L\}$. Hence, $D/E=G/\{\id_L\}\simeq G$ is cyclic by Corollary 1 (p. 71).

(b) Suppose there's an intermediate field $K\varsubsetneq K'\varsubsetneq L$. Let $P'$ be the unique prime in $R'$ lying under $Q$, then $P'$ also lies over $P$. As $P$ is totally ramified in $K'$, we have $e(P'/P)=[K':K]>1$.

Consider the inertia field $L_E$. If it's an intermediate field, then by assumption $P$ is totally ramified in $L_E$, i.e., $1=e(Q_E/P)=[L_E:K]$, a contradiction. Thus we have $L_E=K$ or $L$. However, if $L_E=K$, then $e=[L:K]$. So $P$ is totally ramified in $L$, which is not the case. And if $L_E=L$, then $e=1$. This forces $e(P'/P)=1$, which is also not the case. We may conclude that there's no intermediate field between $K$ and $L$.

Since there's no such $K'$, so by Galois theory we know there's no non-zero proper subgroup in $G$. Let's show $G$ is cyclic of prime order. Let $\id_L\neq\sigma\in G$, then $\langle\sigma\rangle=G$, so $G$ is cyclic. Moreover, if $\#(G)=ab$ where $a,b>1$, then $\langle\sigma^a\rangle$ forms a non-trivial proper subgroup of $G$, which is absurd. So $G$ is of prime order.

(c) Suppose there's an intermediate field $K\varsubsetneq K'\varsubsetneq L$. Let $P'$ be the unique prime in $K'$ lying under $Q$, then by assumption, $P'$ is the unique prime in $K'$ lying over $P$.

Consider the decomposition field $L_D$. If it's an intermediate field, then by assumption $Q_D$ is the only prime in $L_D$ lying over $P$, so $[L_D:K]=e(Q_D/P)f(Q_D/P)=1$, a contradiction. Thus we have $L_D=K$ or $L$. However, if $L_D=K$, then $r=1$. So $Q$ is the only prime in $L$ lying over $P$, which is not the case. And if $L_D=L$, then $e=f=1$. These force $e(P'/P)=f(P'/P)=1$ and so $[K':K]=1\cdot 1=1$, which is also not the case. We may conclude that there's no intermediate field between $K$ and $L$.

Again, since there's no such $K'$, so by Galois theory we know there's no non-zero proper subgroup in $G$. This means $G$ is cyclic of prime order.

(d) First note that if there are no intermediate fields $K\varsubsetneq K'\varsubsetneq L$, then by Galois theory, the only subgroups of $G$ are $\{\id_L\}$ and $G$. In this case, take $H=G$ and done.

Now, assume there's at least one intermediate field $K'$. Then for each such $K'$, by assumption we have $e(P'/P)=1$ for each $P'$ in $K'$ lying over $P$. So by Theorem 29(3), $K'\subseteq L_E$. This means $L_E$ contains all intermediate fields. Moreover, since $P$ is ramified in $L$. So $L_E\varsubsetneq L$. This shows $L_E$ is the largest intermediate field. Hence by Galois theory, the corresponding subgroup $E$ in $G$ is the smallest non-trivial subgroup in $G$. (The uniqueness is clear.)

Let's show $G$ and $H=E$ have the desired properties. If there are distinct prime divisors $p,q$ of $\#(G)$, then by Cauchy's theorem, $\exists \sigma,\tau\in G$ with order $p,q$, respectively. Since $H$ is the smallest subgroup, $H\subseteq \langle\sigma\rangle,\langle\tau\rangle$. This implies $\#(H)\mid p,q$ and so $\#(H)=1$, which is absurd. So $\#(G)$ is a prime power. Consequently, $H$ must have prime order. Lastly, $H\subset Z(G)$ is clear.

(e) Similar to (d), we assume there's at least one intermediate field $K'$. For each such $K'$, by assumption we have $e(P'/P)=f(P'/P)=1$ for each $P'$ in $K'$ lying over $P$. So by Theorem 29(1), $K'\subseteq L_D$. This means $L_D$ contains all intermediate fields. Moreover, since $P$ does not split completely in $L$. So $L_D\varsubsetneq L$. This shows $L_D$ is the largest intermediate field. Hence by Galois theory, the corresponding subgroup $D$ in $G$ is the smallest non-trivial subgroup in $G$. (The uniqueness is clear.)

As an example, let $K=\QQ$ and $L=\QQ(e^{2\pi i/5})$. We know $G=(\ZZ/5\ZZ)^\times\simeq C_4$, the cyclic group of order $4$. There's only one subgroup in $C_4$, namely, $\{0,2\}$. So by Galois correspondence, there's only one intermediate field. Moreover, by Exercise 2.8, we know this field is $\QQ(\sqrt{5})$. Note that the corresponding subgroup $H$ is isomorphic to $\{0,2\}$, which clearly has the desired properties.

Consider $p=11$. Since $5$ is a square mod $11$, by Theorem 25 (p. 52), we have $11$ splits completely in $\QQ(\sqrt{5})$. Moreover, by Theorem 26 (p. 53), $f$ is the order of $11$ in $(\ZZ/5\ZZ)^\times$, which is $2$. So $11$ does not split completely in $L=\QQ(e^{2\pi i/5})$.

(f) Suppose $P$ is inert in every intermediate field $K'$. Note the condition on $L$ implies that $r>1$ or $e>1$. If $r>1$, then the assumption in (c) is satisfied. So $G$ is cyclic of prime order.

Assume $e>1$, then the assumption in (d) is satisfied. So $\#(G)=p^m$. And $E$ is the unique smallest non-trivial subgroup in $G$. Note that if $K\varsubsetneq L_D$, then $L_D$ is an intermediate field. By assumption $P$ is inert in $L_D$. So $1<[L_D:K]=e(Q_D/P)f(Q_D/P)=1$, a contradiction. This shows $K=L_D$ and so $G=D$.

From (d) we know $E=H\subseteq Z(G)\subseteq G=D$. By the third isomorphism theorem, $G/Z(G) \simeq (G/H)/(Z(G)/H) = (D/E)/(Z(G)/H)$. From Corollary 1 we know  $D/E$ is cyclic. In particular, so is $(D/E)/(Z(G)/H)\simeq G/Z(G)$. Thus by group theory we have $G$ is abelian. And the fundamental theorem of finite abelian groups tells us $G$ is a direct sum of cyclic groups of prime power order. Note that if $G\simeq \ZZ/p^{m_1}\ZZ\oplus\cdots\oplus\ZZ/p^{m_k}\ZZ$ where $k\geq 2$, then it would contradict to the uniqueness of $H$, because in this case there are $k$ subgroups of order $p$. So we must have $G\simeq \ZZ/p^m\ZZ$ is cyclic.

\begin{comment}
(b) Suppose there's an intermediate field $K\varsubsetneq K'\varsubsetneq L$. Let $P'$ be the unique prime in $R'$ lying under $Q$, then $P'$ also lies over $P$. Since $P$ is not totally ramified in $L$, we have $e<n$. So $n=ref<rnf$ and thus $rf>1$.

On the other hand, since $P$ is totally ramified in $K'$, we have $e(P'/P)=[K':K]$. So $e=e(Q/P)=e(Q/P')e(P'/P)=e'\cdot[K':K]$. Moreover, the diagram in page 74 tells us $ref=r'e'f'\cdot[K':K]>r'e'f'$. So $er'f'>r'e'f'$ and thus $e>e'\geq 1$. From this together with $rf>1$, we know that $K\varsubsetneq L_E\varsubsetneq L$. So by assumption, $e(Q_E/P)=[L_E:K]$ and hence $1=rf$, a contradiction.

Since there's no such $K'$, so by Galois theory we know there's no non-zero proper subgroup in $G$. Let's show $G$ is cyclic of prime order. Let $\id_L\neq\sigma\in G$, then $\langle\sigma\rangle=G$, so $G$ is cyclic. Moreover, if $\#(G)=ab$ where $a,b>1$, then $\langle\sigma^a\rangle$ forms a non-trivial proper subgroup of $G$, which is absurd. So $G$ is of prime order.

(c) Suppose there's an intermediate field $K\varsubsetneq K'\varsubsetneq L$. Let $P'$ be the unique prime in $R'$ lying under $Q$, then by assumption, $P'$ is the unique prime lying over $P$. Since $P$ (resp. $P'$) splits into $r$ (resp. $r'$) primes in $S$, we have $r=r'$. Also note that by the assumption on $L$ we know $r>1$.

Similar to (b), we have $ref=r'e'f'\cdot[K':K]>r'e'f'$. So $ef>e'f'\geq 1$. From this together with $r>1$, we know that $K\varsubsetneq L_D\varsubsetneq L$. By assumption, $Q_D$ is the only prime in $S_D$ lying over $P$. So $[L_D:K]=e(Q_D/P)f(Q_D/P)$ and hence $r=1$, a contradiction.

Again, since there's no such $K'$, so by Galois theory we know there's no non-zero proper subgroup in $G$. This means $G$ is cyclic of prime order.

(d) First note that if there are no intermediate fields $K\varsubsetneq K'\varsubsetneq L$, then by Galois theory, the only subgroups of $G$ are $\{\id_L\}$ and $G$. In this case, take $H=G$ and done.

Now, assume there's at least one intermediate field $K'$. Note that since $G$ is a finite group. So we know there are only finitely many such $K'$'s. Take $K'$ be the compositum of all intermediate fields, and $H$ be the corresponding subgroup in $G$. Since by assumption $P$ is unramified in every intermediate field. By Theorem 31 (p. 76), $P$ is unramified in $K'$. Thus we have $K'\varsubsetneq L$. This implies $H$ is non-trivial. Moreover, by the selection of $K'$, it's easy to see that $H$ is the unique smallest subgroup in $G$.

It's remaining to show that $H$ is normal in $G$. Note that for each $\sigma\in G$, $\sigma H\sigma^{-1}$ forms another subgroup with $\#(\sigma H\sigma^{-1})=\#(H)$. And by the uniqueness, we have $\sigma H\sigma^{-1}=H$. This means $H$ is normal in $G$.

Let's show the groups have the desired properties. If there are distinct prime divisors $p,q$ of $\#(G)$, then by Cauchy's theorem, $\exists \sigma,\tau\in G$ with order $p,q$, respectively. Since $H$ is the smallest subgroup, $H\subseteq \langle\sigma\rangle,\langle\tau\rangle$. This implies $\#(H)\mid p,q$ and so $\#(H)=1$, which is absurd. So $\#(G)$ is a prime power. Consequently, $H$ must have prime order. Lastly, $H\subset Z(G)$ is clear.

(e) Similar to (d), we assume there's at least one intermediate field $K'$. And let $K'$ be the compositum of all intermediate fields, $H$ be the corresponding subgroup in $G$. Since this time $P$ splits completely in every intermediate field. By Theorem 31 again, $P$ splits completely in $K'$. Thus we have $K'\varsubsetneq L$. The remaining arguments are exactly the same in (d).

As an example, let $K=\QQ$ and $L=\QQ(e^{2\pi i/5})$. We know $G=(\ZZ/5\ZZ)^\times\simeq C_4$, the cyclic group of order $4$. There's only one subgroup in $C_4$, namely, $\{0,2\}$. So by Galois correspondence, there's only one intermediate field. Moreover, by Exercise 2.8, we know this field is $\QQ(\sqrt{5})$. Note that the corresponding subgroup $H$ is isomorphic to $\{0,2\}$, which clearly has the desired properties.

Consider $p=11$. Since $5$ is a square mod $11$, by Theorem 25 (p. 52), we have $11$ splits completely in $\QQ(\sqrt{5})$. Moreover, by Theorem 26 (p. 53), $f$ is the order of $11$ in $(\ZZ/5\ZZ)^\times$, which is $2$. So $11$ does not split completely in $L=\QQ(e^{2\pi i/5})$.
\end{comment}

\subsection*{Exercise 4.6}

By Exercise 2.42, $K$ contains $\QQ(\sqrt{k})$ where $k:=mn/\gcd(m,n)^2$. This is the third quadratic subfield.

Note that since $G$ is abelian, by the remark in page 76, we know the groups $D$ and $E$ are independent of the primes in $K$ lying over $p$. Also note that the corresponding fields $K_D$ and $K_E$ have only five candidates, namely, $\QQ,\QQ(\sqrt{m}),\QQ(\sqrt{n}),\QQ(\sqrt{k})$ and $K$.

The main results we are going to use are Theorem 28 (p. 70) and Theorem 29 (p. 73). All the splittings of a given prime $p$ in the quadratic fields can be found in Theorem 25 (p. 52). In addition, we know $4=[K:\QQ]=ref$.

(a) Since $p$ is ramified in each of the quadratic subfields, so $p$ splits into a square of single prime in these fields. By Theorem 29(3), $K_E$ cannot be any of these quadratic subfields, let alone $K$. This implies $K_E=\QQ$. So $e=4$ and $f=r=1$, i.e., $p$ splits into $P^4$ in $K$.

As an example, consider $n=2,m=3$ and $p=2$, then $2$ is ramified in $\QQ(\sqrt{2}),\QQ(\sqrt{3})$ and $\QQ(\sqrt{6})$. So $2$ splits into $P^4$ in $K=\QQ(\sqrt{2},\sqrt{3})$.

(b) Suppose $p$ splits completely in each of the quadratic subfields, then by Theorem 29(1), $K_D$ contains all these fields. This implies $K_D=K$. So $r=4$ and $e=f=1$, i.e., $p$ splits completely in $K$.

As an example, consider $n=-2,m=7$ and $p=3$, then $3$ splits completely in $\QQ(\sqrt{-2}),$ $\QQ(\sqrt{7})$ and $\QQ(\sqrt{-14})$. So $3$ splits completely in $K=\QQ(\sqrt{-2},\sqrt{7})$.

(c) Suppose $p$ is inert in each of the quadratic subfields, this forces $f\geq 2$. Moreover, by Theorem 29(3), $K_E$ contains all these fields. This implies $K_E=K$ and so $e=1$.

If $f=2$, then $r=2$, $K_D$ is a quadratic field. So $p$ is inert in $K_D$, which is impossible because we know the corresponding ramification index and inertial degree must be $1$. On the other hand, if $f=4$, then $r=1=e$, $p$ is inert in $K$. By Exercise 4.5(a), we have $G$ is cyclic, which is a contradiction. So we may conclude that this case can't occur.

(d) For $p$ splits into $PQ$ in $K$, this means $r=f=2$ and $e=1$. Take $n=2,m=7$ and $p=3$. Since $3$ is inert in $\QQ(\sqrt{2})$, we have $f\geq 2$. And since $3$ splits into two primes in $\QQ(\sqrt{7})$, we have $r\geq 2$. From these two we may conclude that in fact $r=f=2$ and $e=1$.

For $p$ splits into $(PQ)^2$ in $K$, this means $r=e=2$ and $f=1$. Take $n=2,m=-7$ and $p=2$. Since $2$ splits into a square in $\QQ(\sqrt{2})$, we have $e\geq 2$. And since $2$ splits into two primes in $\QQ(\sqrt{-7})$, we have $r\geq 2$. From these two we may conclude that in fact $r=e=2$ and $f=1$.

For $p$ splits into $P^2$ in $K$, this means $e=f=2$ and $r=1$. Take $n=3,m=5$ and $p=2$. Since $2$ splits into a square in $\QQ(\sqrt{3})$, we have $e\geq 2$. And since $2$ is inert in $\QQ(\sqrt{5})$, we have $f\geq 2$. From these two we may conclude that in fact $e=f=2$ and $r=1$.

\subsection*{Exercise 4.7}

If $K=\QQ(\sqrt{m})$ and $L=\QQ(\sqrt{n})$ where $m,n\neq 1$ are distinct square-free integers, then $KL=\QQ(\sqrt{m},\sqrt{n})$, which is normal over $\QQ$ with degree $4$. And from Exercise 4.6 we know there's another quadratic subfield $\QQ(\sqrt{k})$ where $k:=mn/\gcd(m,n)^2$ contained in $KL$.

All the splittings of a given prime $p$ in the quadratic fields can be found in Theorem 25 (p. 52). In addition, the numbers $r,e,f$ are defined corresponding to the splitting of $p$ in $KL$.

(a) Take $K=\QQ(\sqrt{-1}),L=\QQ(\sqrt{3})$ and $p=2$. Then $2$ is totally ramified in both $K$ and $L$. Note that since $2$ is inert in $\QQ(\sqrt{k})=\QQ(\sqrt{-3})$, we know $f\geq 2$. So $2$ cannot be totally ramified in $KL$.

(b) Take $K=\QQ(\sqrt{-1}),L=\QQ(\sqrt{7})$ and $p=2$. Then $2$ splits into a square of single prime in both $K$ and $L$. Note that since $2$ splits into two primes in $\QQ(\sqrt{k})=\QQ(\sqrt{-7})$, we know $r\geq 2$. So $KL$ cannot contain a unique prime lying over $2$.

(c) Take $K=\QQ(\sqrt{-3}),L=\QQ(\sqrt{5})$ and $p=2$. Then $2$ is inert in both $K$ and $L$. Note that since $2$ splits into two primes in $\QQ(\sqrt{k})=\QQ(\sqrt{-15})$, we know $r\geq 2$. So $2$ cannot be inert in $KL$.

(d) Take $K=\QQ(\sqrt{-1}),L=\QQ(\sqrt{3})$ and $p=2$. Then $2$ is totally ramified in both $K$ and $L$, so the corresponding inertial degrees are all $1$. Note that since $2$ is inert in $\QQ(\sqrt{k})=\QQ(\sqrt{-3})$, we know $f\geq 2$. So the residue field extension of $\ZZ/2\ZZ$ is not trivial for $KL$.

\subsection*{Exercise 4.8 \color{red}(incomplete, missing (d) and (e))}

For the followings, let $r,e,f$ be fixed.

(a) First note that if $p,q$ are primes and $p\neq q$, write $q=p^0\cdot q$. Then by Theorem 26 (p. 53), the ramification index (of the splitting of $p$ in the $q$-th cyclotomic field $K$) is $1$.

Write $(\ZZ/q\ZZ)^\times=\langle a\rangle$, then $\ord(a^r)=k=(q-1)/r=[K:\QQ]/r$ in $(\ZZ/q\ZZ)^\times$. Since $\gcd(a^r,q)=1$, we may again use Dirichlet theorem to find a prime $p$ of the form $ql+a^r$. This means $p\equiv a^r \pmod{q}$ and so they have the same order. By Theorem 26 again, the inertial degree is $[K:\QQ]/r$. Moreover, the ramification index is clearly $1$ by the above observation. These two together imply that this particular $p$ splits into $[K:\QQ]\cdot([K:\QQ]/r)^{-1}=r$ distinct primes in $K$.

(b) Similar to (a), but this time we find a prime $q$ of the form $(rf)k+1$. Write $(\ZZ/q\ZZ)^\times=\langle a\rangle$, then $\ord(a^r)=fk=(q-1)/r=[K:\QQ]/r$ in $(\ZZ/q\ZZ)^\times$. Take a prime $p$ of the form $ql+a^r$, then the inertial degree is $\ord(p)=\ord(a^r)=[K:\QQ]/r$ in $(\ZZ/q\ZZ)^\times$. And the ramification index is $1$. So $p$ splits into $r$ distinct primes in $K$.

We claim that $K$ contains a subfield of degree $rf$ over $\QQ$. It's equivalent to find a subgroup $H$ in $G:=(\ZZ/q\ZZ)^\times$ with $[G:H]=rf$. This is easy because $G=\langle a\rangle$, so we may take $H=\langle a^{rf}\rangle=\{a^{rf},a^{2rf},\ldots,a^{krf}=a^{q-1}=1\}$, then $[G:H]=rfk/k=rf$.

Since $p$ splits into $r$ distinct primes in $K$. By the remark of Corollary 2 (p. 72), $K_D$ is the unique intermediate field with $[K_D:\QQ]=r$. So $D$ is the unique subgroup of $G$ with $[G:D]=r$. This implies $D=\langle a^r\rangle=\{a^r,a^{2r},\ldots,a^{fkr}=1\}\supset H$. So the fixed field of $H$ contains $K_D$ and hence $p$ splits into $r$ distinct primes in this field by the same remark.

(c) Take a prime $q$ of the form $(ref)k+1$. Write $(\ZZ/q\ZZ)^\times=\langle a\rangle$, then $\ord(a^r)=efk=(q-1)/r=[K:\QQ]/r$ in $(\ZZ/q\ZZ)^\times$. We need to show that it's possible to find a prime $p$ of the form $ql+a^r$ and $el'+1$, and check that this particular $p$ has all the desired properties.

First note that $\gcd(q,e)=\gcd(refk+1,e)=\gcd(1,e)=1$. So by Chinese remainder theorem, we can find $x\in\ZZ$ s.t. $x\equiv a^r\pmod{q}$ and $x\equiv 1\pmod{e}$. In fact, if we wrtie $1=uq+ve$ where $u,v\in\ZZ$, then we may take $x=uq+a^rve$. We next claim that $\gcd(x,qe)=1$. (Consequently, we may find a prime $p$ of the form $qel''+x$.)

If there is a prime $d\mid\gcd(x,qe)=\gcd(uq+a^rve,qe)=\gcd(1-ve+a^rve,qe)$, then we know $d\mid qe$. If $d\mid q$, then $d=q$. So $a^r\equiv x\equiv 0 \pmod{q}$, which is absurd. And if $d\mid e$, then since $d\mid 1-ve+a^rve$, so $d\mid 1$, which is also impossible. This shows that $\gcd(x,qe)=1$.

Take a prime $p$ of the form $qel''+x$. Clearly we have $p\equiv 1\pmod{e}$. Moreover, since $p\equiv a^r\pmod{q}$, the inertial degree is $\ord(p)=\ord(a^r)=[K:\QQ]/r$ in $(\ZZ/q\ZZ)^\times$. And the ramification index is $1$. So $p$ splits into $r$ distinct primes in $K$.

Lastly, we check that $K$ contains a subfield of degree $rf$ over $\QQ$. Similar to (b), we find that $H=\langle a^{rf}\rangle=\{a^{rf},a^{2rf},\ldots,a^{ekrf}=a^{q-1}=1\}$ is a subgroup with $[G:H]=refk/ek=rf$, so the fixed field of $H$ is a subfield of degree $rf$ over $\QQ$.

%(e) We follow the procedure in (c). First find a prime $q$ of the form $30k+1$. We can take $k=1$ and so $q=31$. Note that $(\ZZ/31\ZZ)^\times=\langle3\rangle$. In our notation $a=3$. Next, write $1=1\cdot31+(-15)\cdot2$ and set $x=1\cdot31+3^5\cdot(-15)\cdot2=-7259$. Finally, take a prime $p$ of the form $31\cdot2\cdot l''-7259=62l''-7259$. A simple calculation shows that when $l''=120$, $p=181$ is a prime.

\subsection*{Exercise 4.9}

(a) For the decomposition groups, $D' = D(Q'/P) = D(\sigma(Q)/P)$, and
\begin{align*}
    D(\sigma(Q)/P) & = \{\tau\in G\mid \tau\sigma(Q)=\sigma(Q)\} = \{\tau\in G\mid \sigma^{-1}\tau\sigma(Q)=Q\} \\
    &= \{\tau\in G\mid \sigma^{-1}\tau\sigma \in D(Q/P)\} = \sigma D(Q/P) \sigma^{-1} = \sigma D\sigma^{-1}
\end{align*}
On the other hand, for the inertia groups, $E' = E(Q'/P) = E(\sigma(Q)/P)$, and
\begin{align*}
    E(\sigma(Q)/P) &= \{\tau\in G\mid \tau(\alpha)\equiv\alpha \pmod{\sigma(Q)}, \forall\alpha\in S\} \\
    &= \{\tau\in G\mid \tau(\alpha)-\alpha\in\sigma(Q), \forall\alpha\in S\} \\
    &= \{\tau\in G\mid \sigma^{-1}\tau(\alpha)-\sigma^{-1}(\alpha)\in Q, \forall\alpha\in S\} \\
    &= \{\tau\in G\mid \sigma^{-1}\tau\sigma\sigma^{-1}(\alpha)-\sigma^{-1}(\alpha)\in Q, \forall\alpha\in S\} \\
    &= \{\tau\in G\mid \sigma^{-1}\tau\sigma(\beta) \equiv \beta \pmod{Q}, \forall\beta\in S\} \\
    &= \{\tau\in G\mid \sigma^{-1}\tau\sigma\in E(Q/P)\} \\
    &= \sigma E(Q/P)\sigma^{-1} = \sigma E\sigma^{-1}
\end{align*}

(b) By the uniqueness of Frobenius automorphisms (Theorem 32 (p. 77)), it's sufficient to show that $\sigma\phi\sigma^{-1}$ has the same property with $\phi'$, i.e., $\sigma\phi\sigma^{-1}(\alpha)\equiv\alpha^{\|P\|} \pmod{Q'=\sigma(Q)}$ for all $\alpha\in S$. But notice that
\begin{align*}
    &\sigma\phi\sigma^{-1}(\alpha)\equiv\alpha^{\|P\|} \pmod{\sigma(Q)}, \forall\alpha\in S \\
    \iff{} &\sigma\phi\sigma^{-1}(\alpha)-\alpha^{\|P\|}\in\sigma(Q), \forall\alpha\in S \\
    \iff{} &\sigma^{-1}\sigma\phi\sigma^{-1}(\alpha)-\sigma^{-1}(\alpha^{\|P\|}) \in Q, \forall\alpha\in S \\
    \iff{} &\phi\sigma^{-1}(\alpha)-\sigma^{-1}(\alpha)^{\|P\|} \in Q, \forall\alpha\in S \\
    \iff{} &\phi(\beta)-\beta^{\|P\|} \in Q, \forall\beta\in S \\
    \iff{} &\phi(\beta)\equiv\beta^{\|P\|} \pmod{Q}, \forall\beta\in S
\end{align*}
Since the last statement holds, the result follows.

\subsection*{Exercise 4.10}

(a) Let $\sigma\in D(V/Q)\subseteq\Gal(LM/L)$, then $\sigma(V)=V$. Observe that $\sigma|_M(U)=\sigma|_M(V\cap M)=\sigma(V\cap M)=\sigma(V)\cap\sigma(M)=V\cap\sigma|_M(M)=V\cap M=U$, so $\sigma|_M\in D(U/P)$.

(b) Let $\sigma\in E(V/Q)\subseteq\Gal(LM/L)$, then $\sigma(\alpha)\equiv\alpha \pmod{V}$ for all $\alpha\in LM\cap\mathbb{A}$. We need to claim that $\sigma|_M(\alpha)\equiv\alpha\pmod{U}$ for all $\alpha\in M\cap\mathbb{A}$. Fix such $\alpha$, we know $\sigma|_M(\alpha)-\alpha\in M$. On the other hand, since $\alpha\in M\cap\mathbb{A}\subset LM\cap\mathbb{A}$, we have $\sigma|_M(\alpha)-\alpha=\sigma(\alpha)-\alpha\in V$. So $\sigma|_M(\alpha)-\alpha\in M\cap V=U$ and hence $\sigma|_M(\alpha)\equiv\alpha\pmod{U}$.

(c) Let $Q$ be any prime in $L$ lying over $P$ and $V$ be any prime in $LM$ lying over $Q$. Since $e(U/P)=1$, by Theorem 28 (p. 70), we have $E(U/P)=\{\id_M\}$. By (b) we know $E(V/Q)|_M:=\{\sigma|_M \mid \sigma\in E(V/Q)\} \subseteq E(U/P)=\{\id_M\}$, so $\sigma|_M=\id_M$ for all $\sigma\in E(V/Q)$. This means $M$ is fixed by all elements in $E(V/Q)$. Moreover, since $L$ is necessarily fixed by these elements, we have $E(V/Q)=\id_{LM}$, which implies $e(V/Q)=1$.

(d) Similar to (c), since $e(U/P)=f(U/P)=1$, by Theorem 28 again, we have $D(U/P)=\{\id_M\}$. By (a) we know $D(V/Q)|_M:=\{\sigma|_M \mid \sigma\in D(V/Q)\} \subseteq D(U/P)=\{\id_M\}$, so $\sigma|_M=\id_M$ for all $\sigma\in D(V/Q)$. This means $M$ is fixed by all elements in $D(V/Q)$. Moreover, since $L$ is necessarily fixed by these elements, we have $D(V/Q)=\id_{LM}$, which implies $e(V/Q)=f(V/Q)=1$.

(e) Since $P$ is unramified in $M$, we know $D(U/P)$ is isomorphic to the Galois group of $(M\cap\mathbb{A})/U$ over $(K\cap\mathbb{A})/P$. So if two maps $\phi(V/Q)|_M$ and $\phi(U/P)^{f(Q/P)}$ in $D(U/P)$ induce the same map in the Galois group, then they are necessarily the same. 

Let $\ovl{\phi}(V/Q)|_M$ be the map induced from $\phi(V/Q)|_M$. Note that $\ovl{\phi}(V/Q)$ generates the Galois group of $(LM\cap\mathbb{A})/V$ over $(L\cap\mathbb{A})/Q$ and maps an element in $(LM\cap\mathbb{A})/V$ to its $\|Q\|$ power. Now, viewing $(LM\cap\mathbb{A})/V$ as a field extension of $(M\cap\mathbb{A})/U$. Given $x+U\in(M\cap\mathbb{A})/U$, then $$\ovl{\phi}(V/Q)|_M(x+U)=\ovl{\phi}(V/Q)(x+U)=(x+U)^{\|Q\|}=x^{\|Q\|}+U$$

On the other hand, let $\ovl{\phi}(U/P)$ be the map induced from $\phi(U/P)$. Note that $\ovl{\phi}(U/P)$ generates the Galois group of $(M\cap\mathbb{A})/U$ over $(K\cap\mathbb{A})/P$ and $\ovl{\phi}(U/P)(x+U)=(x+U)^{\|P\|}=x^{\|P\|}+U$. So $$\ovl{\phi}(U/P)^{f(Q/P)}(x+U)=x^{\|P\|^{f(Q/P)}}+U$$

Consider the field extension $(L\cap\mathbb{A})/Q$ over $(K\cap\mathbb{A})/P$ with degree $f(Q/P)$, then we have $\|Q\|=\|P\|^{f(Q/P)}$. This implies that $\ovl{\phi}(V/Q)|_M=\ovl{\phi}(U/P)^{f(Q/P)}$, as desired.

\subsection*{Exercise 4.11}

(a) Suppose $K\subset L\subset M$ and $M$ is normal over $K$, then $LM=M$. Put $V=U$ in this case. And by Exercise 4.10(e), we have $\phi(V/Q)|_M=\phi(U/Q)=\phi(U/P)^{f(Q/P)}$.

(b) By the uniqueness of Frobenius automorphisms (Theorem 32 (p. 77)), it's sufficient to show that $\phi(U/P)|_L$ has the same property with $\phi(Q/P)$, i.e., $\phi(U/P)|_L(\alpha)\equiv\alpha^{\|P\|} \pmod{Q}$ for all $\alpha\in L\cap\mathbb{A}$. Fix such $\alpha$, we know $\phi(U/P)|_L(\alpha)-\alpha^{\|P\|}\in L$. On the other hand, since $\alpha\in L\cap\mathbb{A}\subset M\cap\mathbb{A}$, we have $\phi(U/P)|_L(\alpha)-\alpha^{\|P\|}=\phi(U/P)(\alpha)-\alpha^{\|P\|}\in U$. So $\phi(U/P)|_L(\alpha)-\alpha^{\|P\|} \in L\cap U=Q$ and hence $\phi(U/P)|_L(\alpha)\equiv\alpha^{\|P\|} \pmod{Q}$.

\subsection*{Exercise 4.12}

(a) Since $K/\QQ$ is normal and $p$ is unramified in $K$ (as $p$ is unramified in $\QQ(\omega)$), we can consider the Frobenius automorphism $\phi(P/p)$. Let $Q$ be a prime in $\QQ(\omega)$ lying over $P$. From Exercise 4.11(b), we know $\phi(P/p)=\phi(Q/p)|_K$. Since $f(P/p)$ is the order of $\phi(P/p)$, we have $f(P/p)$ is the least positive integer s.t. $\phi(P/p)^{f(P/p)}=\id_K$. So $f(P/p)$ is the least positive integer s.t. $\phi(Q/p)^{f(P/p)}$ fixes $K$ pointwise, which means $\phi(Q/p)^{f(P/p)}\in\Gal(L/K)$.

Note that $\phi(Q/p)$ is the automorphism which maps $\omega$ to $\omega^p$ (see p. 77), so $\phi(Q/p)$ corresponds to the element $p\in(\ZZ/m\ZZ)^\times$. Moreover, $\Gal(L/K)\simeq H$, so we know $f(P/p)$ is the least positive integer s.t. $\ovl{p^{f(P/p)}}\in H$, which by definition is $f$. So $f=f(P/p)$.

Alternatively, we can use Theorem 33 (p. 78) to obtain the same result. Consider the action of $p$ on the right coset space of $H$ in $G$. The orbit of $H$ is $\{H,Hp,\ldots,Hp^{f-1}\}$. So by Theorem 33, $f(P/p)=f$. (Viewing $H=H\cdot 1$ and identify $1$ as the identity map.)

(b) Let $\sigma\in\Gal(\QQ(\omega)/\QQ)$ where $\sigma(\omega)=\omega^k$, $\gcd(k,m)=1$. We want to find all $k\in(\ZZ/m\ZZ)^\times$ s.t. $\sigma(\omega+\omega^{-1})=\omega^k+\omega^{-k}=\omega+\omega^{-1}$. It's easy to see that $1,-1\in(\ZZ/m\ZZ)^\times$ are two solutions. Moreover, by Exercise 2.35(b), $[\QQ(\omega):\QQ(\omega+\omega^{-1})]=2$, so by Galois correspondence, $\#(H)=2$. Hence we have $H=\{1,-1\}\subset(\ZZ/m\ZZ)^\times$.

Let $f$ be the least positive integer s.t. $p^f \equiv \pm1\pmod{m}$, then by (a), $f(P/p)=f$. Hence $p$ splits into $\phi(m)/(2f)$ primes in $\QQ(\omega+\omega^{-1})$.

(c) Note that $\ovl{p}\in H$ means $f=1$. And by (a), $f(P/p)=f=1$. So $\ovl{p}\in H \iff$ $p$ splits into two primes in $\QQ(\sqrt{d})$. By Theorem 25 (p. 52), if $p$ is odd, this is equivalent to say that $d$ is a square mod $p$. And if $p=2$, this is equivalent to say that $d\equiv 1\pmod{8}$.

\subsection*{Exercise 4.13}

(a) Note that $K:=\QQ(\alpha)$ and $K':=\QQ(i)$ are two intermediate fields, so by Galois theory, $H:=\Gal(L/K)$ and $H':=\Gal(L/K')$ are two subgroups of $G$.
For $H$, we know $\#(H)=[L:K]=2$. Define $\tau$ by $\tau(\alpha)=\alpha$ and $\tau(i)=-i$, then $\tau|_K=\id_K$. From this it's easy to see that $H=\{\id_L,\tau\}$. Note that $\tau$ represents the permutation $(24)$.

On the other hand, we know $\#(H')=[L:K']=4$. Define $\sigma$ by $\sigma(\alpha)=i\alpha$ and $\sigma(i)=i$, then $\sigma|_{K'}=\id_{K'}$ and so $\sigma\in H'$. Since $\ord(\sigma)=4$, we have $H'=\langle\sigma\rangle$. Note that $\sigma$ represents the permutation $(1234)$.

Since $H\cdot H'\subseteq G$ and $\#(G)=8$, we have $G=\{1,\tau,\sigma,\tau\sigma,\sigma^2,\tau\sigma^2,\sigma^3,\tau\sigma^3\}\simeq D_8$, the dihedral group of order $8$.

(b) By the Corollary of Theorem 31 (p. 76), it's enough to show that $p$ is unramified in $K$. To do this, we want to use Corollary 1 of Theorem 24 (p. 51). Let $f(x)=x^4-m$, then $\Nr(f'(\alpha))=\Nr(4\alpha^3)=4^4\Nr(\alpha)^3=-256m^3$. Since we have $p\nmid -256m^3$, so $p$ is unramified in $K$. 

(c) The right coset space of $H=\Gal(L/K)$ in $G$ is $\{H,H\sigma,H\sigma^2,H\sigma^3\}$. And a simple calculation shows that the action of $\phi(Q/p)=\tau$ on this space gives us three orbits: $\{H\},\{H\sigma,H\sigma\tau=H\tau\sigma^3=H\sigma^3\},\{H\sigma^2\}$. (One may use the fact that $(\tau\sigma)^2=(1)$.) So by Theorem 33 (p. 78), $p$ splits into three primes in $K$.

(d) It's easy to check all the cases. If $\phi=1=\id_L$, then there are four orbits, so $p$ splits completely in $K$. If $\phi=\sigma,\sigma^3$, there is only one orbit, so $p$ is inert in $K$. If $\phi=\tau\sigma,\sigma^2,\tau\sigma^3$, there are two orbits. And finally, if $\phi=\tau\sigma^2$, there are three orbits.

\subsection*{Exercise 4.14}

\subsection*{Exercise 4.15}

If there exist distinct automorphisms $\sigma_1,\ldots,\sigma_n$ which are linearly dependent over $F$, then $\exists a_1,\ldots,a_n\in F$ s.t. $\sum a_i\sigma_i=\mathbf{0}$ is a zero map. We may assume $n$ is minimal. Fix an $x\in F$ s.t. $\sigma_1(x)\neq\sigma_n(x)$. And note that for each $y\in F$, we have $0=\sum a_i\sigma_i(xy)=\sum a_i\sigma_i(x)\sigma_i(y)$. So $\sum a_i\sigma_i(x)\sigma_i=\mathbf{0}$. On the other hand, we clearly have $\sum a_i\sigma_1(x)\sigma_i=\mathbf{0}$. Subtracting these two we obtain $\sum a_i(\sigma_i(x)-\sigma_1(x))\sigma_i=\mathbf{0}$. Note that we have $1\leq\#(\text{terms})<n$ because $\sigma_1(x)-\sigma_n(x)\neq 0$ and the first term cancel out. This contradicts to the assumption that $n$ is minimal.

\subsection*{Exercise 4.16}

Suppose $p$ is unramified in $K$, then by the Corollary of Theorem 31 (p. 76), $p$ is unramified in the normal closure $L$ of $K$ over $\QQ$. Since $L$ contains $\QQ(\sqrt{d})$, $p$ is also unramified in $\QQ(\sqrt{d})$. On the other hand, let $p,q_1,\ldots,q_r$ be all prime numbers dividing $d$ exactly odd times, then $\QQ(\sqrt{d})=\QQ(\sqrt{pq_1\cdots q_n})$. By Theorem 25 (p. 52), $p$ splits into a square in $\QQ(\sqrt{d})$, a contradiction.

Suppose now $d:=\disc(\alpha_1,\ldots,\alpha_n)$ which is divisible by $p$ exactly odd times. Let $\beta_1,\ldots,\beta_n$ be an integral basis, then by Theorem 9 (p. 21), $\beta_i=(m_{i1}\alpha_1+\cdots+m_{in}\alpha_n)/d$ for some $m_{i1},\ldots,m_{in}\in\ZZ$ for each $i$. By applying each embedding of $K$ in $\CC$ to each $\beta_i$, writing as matrices, taking determinants and squaring, we obtain $$\disc(R)=\disc(\beta_1,\ldots,\beta_n)=\disc(\alpha_1,\ldots,\alpha_n)\cdot\jk^2=d\cdot\jk^2$$ where $\jk\in\QQ$. Since $p$ divides $d$ exactly odd times, we must have $p\mid\disc(R)$. And since $L$ contains $\QQ(\sqrt{\disc(R)})$, by Theorem 25 again we get a similar contradiction.

\subsection*{Exercise 4.17}

\subsection*{Exercise 4.18}

\subsection*{Exercise 4.19}

\subsection*{Exercise 4.20}

\subsection*{Exercise 4.21}

\subsection*{Exercise 4.22}

\subsection*{Exercise 4.23}

\subsection*{Exercise 4.24}

\subsection*{Exercise 4.25}

\subsection*{Exercise 4.26}

\subsection*{Exercise 4.27}

\subsection*{Exercise 4.28}

\subsection*{Exercise 4.29}

\subsection*{Exercise 4.30}

\subsection*{Exercise 4.31}

\subsection*{Exercise 4.32}

\subsection*{Exercise 4.33}

\subsection*{Exercise 4.34}

\subsection*{Exercise 4.35}

\subsection*{Exercise 4.36}

\subsection*{Exercise 4.37}

\subsection*{Exercise 4.38}
\phantom{}

\end{document}
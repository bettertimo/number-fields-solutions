\documentclass[../Marcus.tex]{subfiles}

\begin{document}

\chapter{The Ideal Class Group and the Unit Group}

\subsection*{Exercise 5.1}

\begin{align*}
&|\ldots,\sigma(\alpha_i),\ldots,\Rcal\tau(\alpha_i),\Ical\tau(\alpha_i),\ldots|     \\
={} &|\ldots,\sigma(\alpha_i),\ldots,\ovl{\tau}(\alpha_i),\Ical\tau(\alpha_i),\ldots|  
&&(\ovl{\tau}(\alpha_i) = \Ical \tau(\alpha_i) \times (-i) + \Rcal \tau(\alpha_i))       \\
={} &\frac{1}{(2i)^s} |\ldots,\sigma(\alpha_i),\ldots,\ovl{\tau}(\alpha_i),\Ical\tau(\alpha_i)\cdot(2i),\ldots| 
&&(\Ical \tau(\alpha_i) \times (2i))    \\
={} &\frac{1}{(2i)^s} |\ldots,\sigma(\alpha_i),\ldots,\ovl{\tau}(\alpha_i),\tau(\alpha_i),\ldots|  
&&(\tau(\alpha_i) = \ovl{\tau}(\alpha_i) + \Ical\tau(\alpha_i)\cdot(2i)).
\end{align*}

\subsection*{Exercise 5.2}

We let $V,W$ be the matrices formed by taking the $v_i,w_i$ as the rows, respectively. By writing the $v$'s (resp. $w$'s) in terms of the $w$'s (resp. $v$'s) over $\ZZ$, we have $V=CW$ (resp. $W=DV$) where $C,D \in \Mat_n(\ZZ)$. And by taking determinants to the equation $V=CDV$, we have $\det(V) = \det(C)\det(D)\det(V)$. Since $\det(V)\neq0$, this implies $\det(C)\det(D)=1$. And since $C \in \Mat_n(\ZZ)$, we have $\det(C)=\pm1$. Hence $|\det(V)| = |\det(C) \det(W)| = |\det(W)|$.

\subsection*{Exercise 5.3}

By Exercise 2.27 (b), there exists a generating set $\{\beta_1,\ldots,\beta_n\}$ of $\wedge$ and $d_1,\ldots,d_n \in \ZZ$ s.t. $\{d_1\beta_1,\ldots,d_n\beta_n\}$ is a generating set of $M$. Let $B,B'$ be the matrices formed by taking the $\beta_i,d_i\beta_i$ as the rows, respectively. Then $\vol(\RR^n/M) = |\det(B')| = |d_1\cdots d_n| |\det(B)| = \#(\wedge/M) \cdot \vol(\RR^n/\wedge)$.

\subsection*{Exercise 5.4}

Let $A$ be that set. We first check that for any $x=(x_1,\ldots,x_n),y=(y_1,\ldots,y_n) \in A$, their midpoint $(x+y)/2\in A$. Using the given inequality, we see that
\begin{align*}
&\sum_{i=1}^r \left| \frac{x_i+y_i}{2} \right| + 2\sum_{\substack{j=1 \\ \text{odd} }}^{n-1-r} \sqrt{ \left( \frac{x_{r+j}+y_{r+j}}{2} \right)^2 + \left( \frac{x_{r+j+1}+y_{r+j+1}}{2} \right)^2 }       \\
\leq{} &\sum_{i=1}^r \left( \frac{ |x_i| }{2} + \frac{ |y_i| }{2} \right)
+ 2\sum_{\substack{j=1 \\ \text{odd} }}^{n-1-r} \frac{1}{2} \left( \sqrt{x_{r+j}^2+x_{r+j+1}^2} +  \sqrt{y_{r+j}^2+y_{r+j+1}^2} \right)      \\
={} &\frac{1}{2} \left( \sum_{i=1}^r |x_i|
+ 2\sum_{\substack{j=1 \\ \text{odd} }}^{n-1-r}  \sqrt{x_{r+j}^2+x_{r+j+1}^2} 
+ \sum_{i=1}^r |y_i| 
+ 2\sum_{\substack{j=1 \\ \text{odd} }}^{n-1-r} \sqrt{y_{r+j}^2+y_{r+j+1}^2}   \right)   \\
\leq{} &\frac{1}{2}(n+n) = n.
\end{align*}
So indeed $(x+y)/2 \in A$.

In general, we note that the line segment $L := \{(1-t)x+ty \mid 0\leq t \leq 1\}$ joining $x$ and $y$ contains a dense subset $D := \{(1-t)x+ty \mid 0 \leq t = k/2^m \leq 1\}$ (so $\ovl{D} \cap L = L$, i.e., the closure of $D$ in $L$ is $L$). Since we know $D \sbe A$ by the first part of the proof, so $L \sbe \ovl{D} \sbe \ovl{A} = A$.

\subsection*{Exercise 5.5}

$n=1$ is clear. Suppose we have $n^n/n! \geq 2^{n-1}$. We will claim that
$$
\frac{(n+1)^{n+1}}{(n+1)!} = \frac{(n+1)^n}{n!} \geq \frac{2n^n}{n!} \geq 2 \cdot 2^{n-1} = 2^n
$$
is true by showing the inequality $(x+1)^x \geq 2x^x$ holds for $x\geq 1$. That is, $(1+1/x)^x \geq 2$ for $x\geq 1$. Since this is obviously true for $x=1$, we then turn to show that it is increasing from $x=1$. And it's sufficient to show the same thing for $\ln(1+1/x)^x = x\ln(1+1/x)$. But note that
$$
x\ln\left(1+\frac{1}{x}\right)
= \frac{ \ln(1+\frac{1}{x}) }{ \frac{1}{x} }
= \frac{ \ln(1+\frac{1}{x}) -\ln1}{ 1+ \frac{1}{x} -1}
=: \frac{ f(1+\frac{1}{x}) -f(1)}{ (1+ \frac{1}{x}) -1}
$$
is just the slope of the line passing through $(1,f(1))$ and $(1+1/x,f(1+1/x))$ where $f(x)=\ln x$. So as $x$ increases, so is the slope.

Combining the inequality $n^n/n! \geq 2^{n-1}$ with the Corollary 2 of Theorem 37 (pp. 95-96), we have
$$
\sqrt{|\disc(R)|}
\geq \frac{n^n}{n!} \left(\frac{\pi}{4}\right)^s
\geq 2^{n-1} \left(\frac{\pi}{4}\right)^s
= 2^{n-1-2s} \pi^s
= 2^{r-1} \pi^s.
$$
So $|\disc(R)| \geq 4^{r-1}\pi^{2s}$.

In particular, if $R\neq\ZZ \implies n=r+2s>1 \implies r>1$ or $s\geq 1 \implies |\disc(R)| >1$. This shows Corollary 3 of Theorem 37 (p. 96).

\subsection*{Exercise 5.6}

In this exercise, all fields are real quadratic field. So we know $n=2$ and $s=0$ for all $m$.

For $m=2$, we have $\|J\| \leq 1$. So every ideal class contains $R$ and hence $R$ is a PID. The same is also true for $m=3,5$.

For $m=6$, we have $\|J\| \leq 2$. So consider the primes lying over $2$: $2R=(2,\sqrt{6})^2$. This means the possible prime divisors of $J\neq R$ is only $(2,\sqrt{6})$. We check that it is principal (and hence $R$ is a PID).

We look for an element whose norm is $\pm2$. Clearly $2+\sqrt{6}$ works. So $\|(2+\sqrt{6})\| = |\Nr(2+\sqrt{6})| = 2$, which is a prime. This implies $(2+\sqrt{6})$ is a prime. And it lies over $2$ because we know $\|(2+\sqrt{6})\|$ is a power of $p$ where $p$ is the unique prime lying under $(2+\sqrt{6})$. Hence, $(2,\sqrt{6}) = (2+\sqrt{6})$ is principal.

For $m=7$, we have $\|J\| \leq 2$. So consider the primes lying over $2$: $2R=(2,1+\sqrt{7})^2$. This means the possible prime divisors of $J\neq R$ is only $(2,1+\sqrt{7})$. We also check that it is principal.

We look for an element whose norm is $\pm2$. Clearly $3+\sqrt{7}$ works. So $\|(3+\sqrt{7})\| = 2$, which is a prime. This implies $(3+\sqrt{7})$ is a prime lying over $2$. Hence, $(2,1+\sqrt{7}) = (3+\sqrt{7})$ is principal.

For $m=173$, we have $\|J\| \leq 6$. So consider the primes lying over $2,3,5$: $2R,3R,5R$ are all primes, which are principal. So $J$ is principal. A similar situation happens for $m=293,437$.

\subsection*{Exercise 5.7}

For $m=10$, we have $\|J\| \leq 3$. So consider the primes lying over $2,3$:
\begin{alignat*}{2}
2R &= (2,\sqrt{10})^2 =: P_2^2  &&\implies \|P_2\| = 2,    \\
3R &= (3,1+\sqrt{10})(3,1-\sqrt{10}) =: P_3Q_3   &&\implies \|P_3\| = \|Q_3\| = 3.
\end{alignat*}
We claim that the above three primes are not principal. For example, if $P_2 = (\alpha)$ for some $\alpha=a+b\sqrt{10},a,b\in \ZZ$, then $|\Nr(\alpha)| = \|P_2\| = 2$. So we have $a^2-10b^2 = \pm2$. But by considering modulo $5$ this implies $a^2 \equiv 2,3 \pmod{5}$, a contradiction. A similar argument works for the other two primes.

Now, we look for an element whose norm is $\pm6$. Clearly $4+\sqrt{10}$ works. And note that if $P$ is a prime divisor of $(4+\sqrt{10})$, then $\|P\| \mid \|(4+\sqrt{10})\| = 6$. So $P$ must lie over $2,3$. This implies that one of the prime divisors of $(4+\sqrt{10})$ is $P_2$, and the other one (call it $U$) is either $P_3$ or $Q_3$. So if we let $U'$ be the remaining prime, then $P_2U,P_2^2,UU'$ are all principal, which means $\ovl{P_2} \cdot \ovl{U} = \ovl{P_2}^2 = \ovl{U} \cdot \ovl{U'}$, which means $\ovl{P_2} = \ovl{U}= \ovl{U'} = \ovl{P_3} = \ovl{Q_3}$. Hence, $\ZZ[\sqrt{10}]$ has two ideal classes.

\subsection*{Exercise 5.8}

\begin{comment}
For $m=223$, we have $\|J\| \leq 14$. So consider the primes lying over $2,3,5,7,11,13$:
\begin{align*}
2R &= (2,1+\sqrt{223})^2   \\
3R &= (3,1+\sqrt{223})(3,1-\sqrt{223})    \\
5R &= \text{prime}    \\
7R &= \text{prime}    \\
11R &= (11,5+\sqrt{223})(11,5-\sqrt{223})    \\
13R &= \text{ prime}
\end{align*}

We look for an element whose norm is $\pm2$: $15+\sqrt{223}$ works. So $\|(15+\sqrt{223})\| = 2$, which is a prime. This implies $(15+\sqrt{223})$ is a prime lying over $2$. Hence, $(2,1+\sqrt{223}) = (15+\sqrt{223})$ is principal.
\end{comment}

\subsection*{Exercise 5.9}

In this exercise, all fields are imaginary quadratic field. So we know $n=2$ and $s=1$ for all $m$.

For $m=-1$, we have $\|J\| \leq 1$. So every ideal class contains $R$ and hence $R$ is a PID. The same is also true for $m=-2,-3,-7$.

For $m=-11$, we have $\|J\| \leq 2$. So consider the primes lying over $2$: $2R$ is a prime, which is principal. So $J$ is principal and hence $R$ is a PID. A similar situation happens for $m=-19,-43,-67,-163$.

\subsection*{Exercise 5.10}

Suppose $R:=\AA\cap\QQ(\sqrt{m})$ is a PID.

(a) Suppose $m\not\equiv 5 \pmod{8}$. We will show that $m=-1,-2,-7$. Since each prime $P$ lying over $2$ is principal, we may write $P=(\alpha)$. Note that we have $|\Nr(\alpha)| = \|P\| = 2$ whenever $m\not\equiv 5 \pmod{8}$.

Case 1: $m$ is even. Write $\alpha = a+b\sqrt{m},a,b\in\ZZ$. Then $a^2 - mb^2 = 2$. This can only happen when $m=-2$.

Case 2: $m$ is odd and $m\equiv3 \pmod{4}$. Write $\alpha = a+b\sqrt{m},a,b\in\ZZ$. Then $a^2 - mb^2 = 2$. This can only happen when $m=-1$.

Case 3: $m$ is odd and $m\equiv1 \pmod{8}$. Write $\alpha = (a+b\sqrt{m})/2,a,b\in\ZZ,a \equiv b \pmod{2}$. Then $a^2 - mb^2 = 8$. This can only happen when $m=-7$.

(b) \textcolor{red}{(There's a typo in here. Change $m<4p$ to $p<|\disc(R)|/4$.)}\footnote{Cohn, Harvey. \textit{Advanced number theory}. Courier Corporation, 2012, p. 151.} Suppose on the contrary that $m$ is a square mod $p$. Let $P$ be any prime lying over $p$. Note that no matter $p$ divides $m$ or not, we always have $\|P\|=p$. So if we write $P=(\alpha)$, then $|\Nr(\alpha)| = p$.

Case 1: $m\equiv2,3 \pmod{4}$ (so $\disc(R)=4m$ and by assumption, $p<4|m|/4 = |m|$). Write $\alpha = a+b\sqrt{m},a,b\in\ZZ$. Then we have $a^2 - mb^2 = a^2 + |m|b^2 = p$. But as $p<|m|$, this is impossible.

Case 2: $m\equiv1 \pmod{4}$ (so $\disc(R)=m$ and by assumption, $p<|m|/4$). Write $\alpha = (a+b\sqrt{m})/2,a,b\in\ZZ,a \equiv b \pmod{2}$. Then we have $a^2 - mb^2 = a^2 + |m|b^2 = 4p$. But as $4p<|m|$, this is also impossible.

The above two cases show that $m$ is not a square mod $p$.

(c) Suppose $m<-19$, then by (a) we know $m\equiv5 \pmod{8}$. And since $m$ is square-free, we know $m\neq-27$. This means $m\leq-35$. On the other hand, since $m\equiv1 \pmod{4}$, $\disc(R)=m$. So by (b), for any odd prime $p$ with $p<|m|/4$, we have $m$ is not a square mod $p$. Since $|m|\geq35$, so the primes $p=3,5,7$ can be used.

For $p=3$, this means $m \equiv 2 \pmod{3}$. For $p=5$, this means $m \equiv 2,3 \pmod{5}$. And for $p=7$, this means $m \equiv 3,5,6 \pmod{7}$. These three together with $m\equiv5 \pmod{8}$ give us six systems of congruence equations, which correspond to the solutions $m\equiv -43,-67,-163,-403,-547,-667 \pmod{840}$ by Chinese remainder theorem.

(d) For $0>m\geq-19$, we shall by (a) only consider $m=-1,-2,-3,-7,-11,-19$, which have already been verified are PID in Exercise 5.9. For $-19 > m > -2000$, we see by (c) that the candidates are
\begin{align*}
m = &-43,-883,-1723,-67,-907,-1747,-163,-1003,\\
&-1843,-403,-1243,-547,-1387,-667,-1507.
\end{align*}
And by Exercise 5.9 again, $m=-43,-67,-163$ have already been verified are PID.

Now, for the remaining numbers, we use (b): If there exists an odd prime $p<|m|/4$ s.t. $(m/p) = 1$, then that $m$ can be eliminated. And from the proof of (c) we know we should start from $p=11$. We in fact don't need to go too far:
$$
\begin{array}{c|c|c|c|c|c|c}
m&-883&-1723&-907&-1747&-1003&-1843      \\
\hline
p&13&11&13&17&11&11   \\
\hline
m&-403&-1243&-547&-1387&-667&-1507     \\
\hline
p&11&17&11&13&11&13\\
\end{array}
$$

\subsection*{Exercise 5.11}

\subsection*{Exercise 5.12}

\subsection*{Exercise 5.13}

For $m=-14$, we have $\|J\| \leq 4$. So consider the primes lying over $2,3$:
\begin{alignat*}{2}
2R &= (2,\sqrt{-14})^2 =: P_2  &&\implies \|P_2\| = 2,      \\
3R &= (3,1+\sqrt{-14})(3,1-\sqrt{-14}) =: P_3Q_3    &&\implies \|P_3\| = \|Q_3\| = 3.
\end{alignat*}
Since the possible prime divisors of $J\neq R$ are these three primes, and $\|J\| \leq 4$, we see that $\ovl{J}=\ovl{R},\ovl{P_2},\ovl{P_3},\ovl{Q_3}$. So the ideal class group has at most four elements. 

We claim that the above three primes are not principal. For example, if $P_2 = (\alpha)$ for some $\alpha=a+b\sqrt{-14},a,b\in \ZZ$, then $|\Nr(\alpha)| = \|P_2\| = 2$. So we have $a^2+14b^2 = \pm2$. But this is clearly impossible. A similar argument works for the other two primes. So the ideal class group has at least two elements. Moreover, this also implies that $\ovl{P_2}$ has order two, so three can be eliminated.

Note that if $P_2P_3 = (\alpha)$ for some $\alpha=a+b\sqrt{-14},a,b\in \ZZ$, then $|\Nr(\alpha)| = \|P_2\|\|P_3\| = 6$. So we have $a^2+14b^2 = \pm6$. But this is clearly impossible. This means $\ovl{P_2} = \ovl{P_2}^{-1} \neq \ovl{P_3}$. So two can also be eliminated.

It remains to decide it is cyclic four or Klein four. We note that in the Klein four group, any two non-identity elements add up to the third non-identity element. But in our case $\ovl{P_3} \cdot \ovl{Q_3}$ is the identity. This shows that the ideal class group of $\ZZ[\sqrt{-14}]$ is cyclic four.

For $m=-39$, we have $\|J\| \leq 3$. So consider the primes lying over $2,3$:
\begin{alignat*}{2}
2R &= \left(2,\frac{1+\sqrt{-39}}{2}\right) \left(2,\frac{1-\sqrt{-39}}{2}\right)  =: P_2Q_2  &&\implies \|P_2\| = \|Q_2\| = 2,     \\
3R &= (3,\sqrt{-39})^2 =: P_3^2  &&\implies \|P_3\| = 3.
\end{alignat*}
Since the possible prime divisors of $J\neq R$ are these three primes, and $\|J\| \leq 3$, we see that $\ovl{J}=\ovl{R},\ovl{P_2},\ovl{Q_2},\ovl{P_3}$. So the ideal class group has at most four elements.

We claim that the above three primes are not principal. For example, if $P_2 = (\alpha)$ for some $\alpha = (a+b\sqrt{-39})/2,a,b\in\ZZ,a \equiv b \pmod{2}$, then $|\Nr(\alpha)| = \|P_2\| = 2$. So we have $a^2+39b^2 = \pm8$. But this is clearly impossible. A similar argument works for the other two primes. So the ideal class group has at least two elements. Moreover, this also implies that $\ovl{P_3}$ has order two, so three can be eliminated.

Note that if $P_2P_3 = (\alpha)$ for some $\alpha = (a+b\sqrt{m})/2,a,b\in\ZZ,a \equiv b \pmod{2}$, then $|\Nr(\alpha)| = \|P_2\|\|P_3\| = 6$. So we have $a^2+39b^2 = \pm24$. But this is clearly impossible. This means $\ovl{P_3} = \ovl{P_3}^{-1} \neq \ovl{P_2}$. So two can also be eliminated.

It remains to decide it is cyclic four or Klein four. Similar as before, since $\ovl{P_2} \cdot \ovl{Q_2}$ is the identity, so the ideal class group of $\AA \cap \QQ[\sqrt{-39}]$ is cyclic four.

\subsection*{Exercise 5.14}

For $m=-21$, we have $\|J\| \leq 5$. So consider the primes lying over $2,3,5$:
\begin{alignat*}{2}
2R &= (2,1+\sqrt{-21})^2 =: P_2^2  &&\implies \|P_2\| = 2,    \\
3R &= (3,\sqrt{-21})^2 =: P_3^2     &&\implies \|P_3\| = 3,    \\
5R &= (5,2+\sqrt{-21})(5,2-\sqrt{-21}) =: P_5Q_5  &&\implies \|P_5\| = \|Q_5\| = 5.
\end{alignat*}
Since the possible prime divisors of $J\neq R$ are these four primes, and $\|J\| \leq 5$, we see that $\ovl{J}=\ovl{R},\ovl{P_2},\ovl{P_3},\ovl{P_5},\ovl{Q_5}$. So the ideal class group has at most five elements.

We claim that the above four primes are not principal. For example, if $P_2 = (\alpha)$ for some $\alpha=a+b\sqrt{-21},a,b\in \ZZ$, then $|\Nr(\alpha)| = \|P_2\| = 2$. So we have $a^2+21b^2 = \pm2$. But this is clearly impossible. A similar argument works for the other primes. So the ideal class group has at least two elements. Moreover, this also implies that $\ovl{P_2}$ has order two, so three and five can be eliminated.

Note that if $P_2P_3 = (\alpha)$ for some $\alpha=a+b\sqrt{-21},a,b\in \ZZ$, then $|\Nr(\alpha)| = \|P_2\|\|P_3\| = 6$. So we have $a^2+21b^2 = \pm6$. But this is clearly impossible. This means $\ovl{P_2} = \ovl{P_2}^{-1} \neq \ovl{P_3}$. So two can also be eliminated.

It remains to decide it is cyclic four or Klein four. We note that in our case, there're at least two different elements of order two, i.e., $\ovl{P_2}$ and $\ovl{P_3}$. This can only happen in the Klein four group. Hence the ideal class group of $\ZZ[\sqrt{-21}]$ is Klein four.

For $m=-30$, we have $\|J\| \leq 6$. So consider the primes lying over $2,3,5$:
\begin{alignat*}{2}
2R &= (2,\sqrt{-30})^2 =: P_2^2  &&\implies \|P_2\| = 2,   \\
3R &= (3,\sqrt{-30})^2 =: P_3^2  &&\implies \|P_3\| = 3,   \\
5R &= (5,\sqrt{-30})^2 =: P_5^2  &&\implies \|P_5\| = 5.
\end{alignat*}
Since the possible prime divisors of $J\neq R$ are these three primes, and $\|J\| \leq 6$, we see that $\ovl{J}=\ovl{R},\ovl{P_2},\ovl{P_3},\ovl{P_5},\ovl{P_2P_3}$. So the ideal class group has at most five elements.

We claim that the above three primes are not principal. For example, if $P_2 = (\alpha)$ for some $\alpha=a+b\sqrt{-30},a,b\in \ZZ$, then $|\Nr(\alpha)| = \|P_2\| = 2$. So we have $a^2+30b^2 = \pm2$. But this is clearly impossible. A similar argument works for the other primes. So the ideal class group has at least two elements. Moreover, this also implies that $\ovl{P_2}$ has order two, so three and five can be eliminated.

Note that if $P_2P_3 = (\alpha)$ for some $\alpha=a+b\sqrt{-30},a,b\in \ZZ$, then $|\Nr(\alpha)| = \|P_2\|\|P_3\| = 6$. So we have $a^2+30b^2 = \pm6$. But this is clearly impossible. This means $\ovl{P_2} = \ovl{P_2}^{-1} \neq \ovl{P_3}$. So two can also be eliminated.

It remains to decide it is cyclic four or Klein four. Similar as before, since there're at least two different elements of order two, i.e., $\ovl{P_2}$ and $\ovl{P_3}$. So the ideal class group of $\ZZ[\sqrt{-30}]$ is Klein four.

\subsection*{Exercise 5.15}

Let $R:=\AA\cap\QQ(\sqrt{-103})$. We have $\|J\| \leq 6$. So consider the primes lying over $2,3,5$:
\begin{alignat*}{2}
2R &= \left(2,\frac{1+\sqrt{-103}}{2}\right) \left(2,\frac{1-\sqrt{-103}}{2}\right)  =: P_2Q_2    &&\implies \|P_2\| = \|Q_2\| = 2,  \\
3R &= \text{prime}      &&\implies \|3R\| = 3^2, \\
5R &= \text{prime}      &&\implies \|5R\| = 5^2.
\end{alignat*}
Since the possible prime divisors of $J\neq R$ are these four primes, and $\|J\| \leq 6$, we see that $\ovl{J}=\ovl{R},\ovl{P_2},\ovl{Q_2},\ovl{P_2}^2,\ovl{Q_2}^2$. We will claim that these five elements are different.

We claim that $P_2,Q_2$ are not principal. If $P_2 = (\alpha)$ for some $\alpha=(a+b\sqrt{-103})/2,a,b\in \ZZ,a \equiv b\pmod{2}$, then $|\Nr(\alpha)| = \|P_2\| = 2$. So we have $a^2+103b^2 = \pm8$. But this is clearly impossible. A similar argument works for $Q_2$.

We claim that $P_2^2,Q_2^2$ are not principal. If $P_2^2 = (\alpha)$ for some $\alpha=(a+b\sqrt{-103})/2,a,b\in \ZZ,a \equiv b\pmod{2}$, then $|\Nr(\alpha)| = \|P_2\|^2 = 4$. So we have $a^2+103b^2 = \pm16 \implies a=\pm4,b=0 \implies P_2^2 = (\alpha) = (2) = P_2Q_2 \implies P_2=Q_2$, which is absurd. A similar argument works for $Q_2^2$.

The above two claims show that $\ovl{P_2},\ovl{Q_2},\ovl{P_2}^2,\ovl{Q_2}^2$ are non-identity elements. Note that in particular, we have $\ovl{P_2}\neq\ovl{Q_2}$. (Because if $\ovl{P_2}=\ovl{Q_2} \implies \ovl{(2)} = \ovl{P_2}\cdot\ovl{Q_2} = \ovl{P_2}^2$, a contradiction.) And moreover, $\ovl{P_2}\neq\ovl{P_2}^2$ and $\ovl{Q_2}\neq\ovl{Q_2}^2$.

We next claim that $\ovl{Q_2}\neq\ovl{P_2}^2$ (by considering $P_2^3$). If $P_2^3 = (\alpha)$ for some $\alpha=(a+b\sqrt{-103})/2,a,b\in \ZZ,a \equiv b\pmod{2}$, then $|\Nr(\alpha)| = \|P_2\|^3 = 8$. So we have $a^2+103b^2 = \pm32$. But this is clearly impossible. So we know $P_2^3$ is not principal. In particular, this implies $\ovl{Q_2}\neq\ovl{P_2}^2$. (Because if $\ovl{Q_2}=\ovl{P_2}^2 \implies \ovl{(2)} = \ovl{P_2}\cdot\ovl{Q_2} = \ovl{P_2}^3$, a contradiction.) A similar argument shows that $\ovl{P_2}\neq\ovl{Q_2}^2$.

Lastly, we claim that $\ovl{P_2}^2\neq\ovl{Q_2}^2$ (by considering $P_2^4$). If $P_2^4 = (\alpha)$ for some $\alpha=(a+b\sqrt{-103})/2,a,b\in \ZZ,a \equiv b\pmod{2}$, then $|\Nr(\alpha)| = \|P_2\|^4 = 16$. So we have $a^2+103b^2 = \pm64 \implies a=\pm8,b=0 \implies P_2^4 = (\alpha) = (4) = P_2^2Q_2^2 \implies P_2^2=Q_2^2 \implies P_2 = Q_2$, which is absurd. So we know $P_2^4$ is not principal. In particular, this implies $\ovl{P_2}^2\neq\ovl{Q_2}^2$. (Because if $\ovl{P_2}^2=\ovl{Q_2}^2 \implies \ovl{(4)} = \ovl{P_2}^2\cdot\ovl{Q_2}^2 = \ovl{P_2}^4$, a contradiction.)

\subsection*{Exercise 5.16}

\subsection*{Exercise 5.17}

First, consider $R:=\ZZ[\omega]$. In this case $n=6,s=3,\disc(R) = -7^5$ (see Exercise 2.8 (a) if necessary). So we have $\|J\| \leq 4$. Consider the primes lying over $2,3$:
\begin{alignat*}{2}
2R &= P_2Q_2 \text{ with } f_2 = 3 &&\implies \|P_2\| = \|Q_2\| = 2^3,    \\
3R &= \text{ prime with } f_3 = 6 &&\implies \|3R\| = 3^6.
\end{alignat*}
Since the possible prime divisors of $J\neq R$ are these three primes, and $\|J\| \leq 4$, we see that $\ovl{J}=\ovl{R}$. Hence $R=\ZZ[\omega]$ is a PID.

Next, consider $R:=\AA\cap\QQ(\omega+\omega^{-1})=\ZZ[\omega+\omega^{-1}]$ by Exercise 2.35 (f). In this case $n=3,s=0,\disc(R) = 7^2$ by Exercise 2.35 (g). So we have $\|J\| \leq 1$ and hence $R=\ZZ[\omega+\omega^{-1}]$ is a PID.

\subsection*{Exercise 5.18}

We recall in Exercise 4.12 (b) we showed the following: Let $\omega:=e^{2\pi i/m}$. Suppose $p\nmid m$ is a prime and $f$ is the least positive integer s.t. $p^f \equiv \pm1\pmod{m}$. Then $p$ splits into $\phi(m)/(2f)$ primes in $\QQ(\omega+\omega^{-1})$ with inertial degree $f$.

Let $R:=\ZZ[\omega+\omega^{-1}]$. For $m=11$, we have $n=5,s=0,\disc(R) = 11^4$ by Exercise 2.35 (g). So $\|J\| \leq 4$. Consider the primes lying over $2,3$: By the above statement, we see that $f_2=f_3=5$. So $2R,3R$ are both primes in $R$. Hence $R$ is a PID.

For $m=13$, we have $n=6,s=0,\disc(R) = 13^5$ by Exercise 2.35 (g). So $\|J\| \leq 9$. Consider the primes lying over $2,3,5,7$: By the above statement, we see that $f_2=6,f_3=3,f_5=2,f_7=6$. So
\begin{alignat*}{2}
2R &= \text{prime with } f_2 = 6     &&\implies \|2R\| = 2^6,       \\
3R &= P_3Q_3 \text{ with } f_3=3  &&\implies \|P_3\|=\|Q_3\|=3^3,       \\
5R &= P_5Q_5U_5 \text{ with } f_5=2  &&\implies \|P_5\|=\|Q_5\|=\|U_5\|=5^2,     \\
7R &= \text{prime with } f_7 = 6     &&\implies \|7R\| = 7^6.
\end{alignat*}
Since the possible prime divisors of $J\neq R$ are these seven primes, and $\|J\| \leq 9$, we see that $\ovl{J}=\ovl{R}$. Hence $R$ is a PID.

\subsection*{Exercise 5.19}

First, consider $R:=\ZZ[\sqrt[3]{2}]$. In this case $n=3,s=1,\disc(R) = -27\cdot(-2)^2$ by Exercise 2.28 (c). So we have $\|J\| \leq 2$. Consider the primes lying over $2$: $2R=(\sqrt[3]{2})^3$ (see p. 49 if necessary). Hence $R$ is a PID.

Next, consider $R:=\ZZ[\alpha]$ where $\alpha^3=\alpha+1$. In this case $n=3,s=1$ (as $f(x)=x^3-x-1$ has one real root), $\disc(R) = \disc(\alpha) = -4\cdot(-1)^3-27\cdot(-1)^2$ by Exercise 2.28 (c) and (d). So we have $\|J\| \leq 1$ and hence $R$ is a PID.

\subsection*{Exercise 5.20}

Let $R:=\AA\cap\QQ(\alpha)$ where $\alpha^3=\alpha+7$. In this case $n=3,s=1$ (as $f(x)=x^3-x-7$ has one real root). Let's find $\disc(R)$. Note that by Exercise 2.28 (c), $\disc(\alpha) = \disc(1,\alpha,\alpha^2) = -4\cdot(-1)^3-27\cdot(-7)^2 = -1319$, which is square-free (in fact, a prime). So by Exercise 2.27 (e), $\{1,\alpha,\alpha^2\}$ is an integral basis of $R$, i.e., $R=\ZZ[\alpha]$. Hence, $\disc(R) = \disc(\alpha) = -1319$. So $\|J\| \leq 10$.

Consider the primes lying over $2,3,5,7$: By Theorem 27 (p. 55), we factor $f(x)=x^3-x-7$ modulo $2,3,5,7$ and see that
\begin{align*}
2R &= \text{prime},    \\
3R &= \text{prime},    \\
5R &= \text{prime},   \\
7R &= P_7Q_7U_7.
\end{align*}
On the other hand, note that $(7) = (\alpha^3-\alpha) = (\alpha)(\alpha+1)(\alpha-1)$. These three factors must be distinct because we know $7$ splits completely in $R$. Moreover, they must all be prime. This shows that $P_7,Q_7,U_7$ are principal. Hence $R$ is a PID.

\subsection*{Exercise 5.21}

The three embeddings of $K$ into $\CC$ map $\sqrt[3]{m}$ to $\sqrt[3]{m},\sqrt[3]{m}\omega,\sqrt[3]{m}\omega^2$ where $\omega=(-1+\sqrt{-3})/2$. So $\NrKQ(\sqrt[3]{m} + a) = (\sqrt[3]{m}+a)(\sqrt[3]{m}\omega+a)(\sqrt[3]{m}\omega^2+a) = m+a^3$.

\subsection*{Exercise 5.22}

In this exercise, we have $n=3,s=1$. And by the application of Theorem 13 (p. 28), we know $R := \AA \cap \QQ(\sqrt[3]{m}) = \ZZ[\sqrt[3]{m}]$ when $m=3,5,6$. Moreover, recall that if $m = hk^2$ where $h, k$ are relatively prime and square-free and $m \not\equiv \pm1 \pmod{9}$, then $\disc(R) = -27m^2/k^2$ (see Exercise 3.26(a)). Since in our cases $m=3,5,6$ are square-free, so we have $\disc(R) = -27m^2$.

From Exercise 3.26 we know how primes split in $R$. And by Exercise 5.21 we know the formula of norm of some particular elements in $R$:

For $m=3$, we have $\|J\| \leq 4$. So consider the primes lying over $2,3$:
\begin{alignat*}{2}
2R &= P_2Q_2 \text{ with } f(P_2/2) = 1, f(Q_2/2) = 2 &&\implies \|P_2\| = 2, \|Q_2\| = 2^2,    \\
3R &= P_3^3 &&\implies \|P_3\| = 3.
\end{alignat*}
We claim that the above three primes are principal. Since $\|(\sqrt[3]{3}-1)\| = |\Nr(\sqrt[3]{3}-1)| = 2$, so $(\sqrt[3]{3}-1) = P_2$ is principal. And since $2R = P_2Q_2$, so $Q_2$ is also principal.

Since $\|(\sqrt[3]{3})\| = |\Nr(\sqrt[3]{3})| = 3$, so $(\sqrt[3]{3}) = P_3$ is principal.

For $m=5$, we have $\|J\| \leq 7$. So consider the primes lying over $2,3,5,7$:
\begin{alignat*}{2}
2R &= P_2Q_2 \text{ with } f(P_2/2) = 1, f(Q_2/2) = 2 &&\implies \|P_2\| = 2, \|Q_2\| = 2^2,    \\
3R &= P_3^3 &&\implies \|P_3\| = 3,    \\
5R &= P_5^3 &&\implies \|P_5\| = 5,    \\
7R &= \text{prime} &&\implies \|7R\| = 7^3.
\end{alignat*}
We claim that the above four primes are principal. Since $\|(\sqrt[3]{5}-2)\| = |\Nr(\sqrt[3]{5}-2)| = 3$, so $(\sqrt[3]{5}-2) = P_3$ is principal. Since $\|(\sqrt[3]{5}+1)\| = |\Nr(\sqrt[3]{5}+1)| = 6 = 2\cdot 3$, we have $(\sqrt[3]{5}+1) = P_2P_3$. And as $P_3$ is principal, so is $P_2$. And since $2R = P_2Q_2$, so $Q_2$ is also principal.

Since $\|(\sqrt[3]{5})\| = |\Nr(\sqrt[3]{5})| = 5$, so $(\sqrt[3]{5}) = P_5$ is principal.

For $m=6$, we have $\|J\| \leq 8$. So consider the primes lying over $2,3,5,7$:
\begin{alignat*}{2}
2R &= P_2^3 &&\implies \|P_2\| = 2,   \\
3R &= P_3^3 &&\implies \|P_3\| = 3,    \\
5R &= P_5Q_5 \text{ with } f(P_5/5) = 1, f(Q_5/5) = 2 &&\implies \|P_5\| = 5, \|Q_5\| = 5^2,    \\
7R &= P_7Q_7U_7 &&\implies \|P_7\| = \|Q_7\| = \|U_7\| = 7.
\end{alignat*}
We claim that the above seven primes are principal. Since $\|(\sqrt[3]{6}-2)\| = |\Nr(\sqrt[3]{6}-2)| = 2$, so $(\sqrt[3]{6}-2) = P_2$ is principal. Since $\|(\sqrt[3]{6})\| = |\Nr(\sqrt[3]{6})| = 6 = 2\cdot 3$, we have $(\sqrt[3]{6}) = P_2P_3$. And as $P_2$ is principal, so is $P_3$.

Since $\|(\sqrt[3]{6}-1)\| = |\Nr(\sqrt[3]{6}-1)| = 5$, so $(\sqrt[3]{6}-1) = P_5$ is principal. And since $5R = P_5Q_5$, so $Q_5$ is also principal.

Since $\|(\sqrt[3]{6}+1)\| = |\Nr(\sqrt[3]{6}+1)| = 7$, we have $(\sqrt[3]{6}+1) = P_7,Q_7$ or $U_7$. WLOG, assume $P_7=(\sqrt[3]{6}+1)$ is principal. On the other hand, since $\|(\sqrt[3]{6}+2)\| = |\Nr(\sqrt[3]{6}+2)| = 14 = 2\cdot 7$, we have $(\sqrt[3]{6}+2) = P_2P_7,P_2Q_7$ or $P_2U_7$. But note that $P_2P_7$ is impossible because otherwise, $(\sqrt[3]{6}+2) = P_2P_7 \sbe P_7 = (\sqrt[3]{6}+1)$ and so $1 \in P_7$, which is absurd. Hence we have $(\sqrt[3]{6}+2)$ is either $P_2Q_7$ or $P_2U_7$. And since $P_2$ is principal, so either $Q_7$ or $U_7$ is principal. Finally, since $7R = P_7Q_7U_7$, so the remaining one is also principal.

\subsection*{Exercise 5.23}

(a) We omit the tedious calculation. Just remember that the three embeddings of $K$ into $\CC$ map $\alpha$ to $\alpha,\alpha\omega,\alpha\omega^2$ where $\omega=(-1+\sqrt{-3})/2$. And we have $\alpha^3 = m$ and $1+\omega+\omega^2=0$.

(b) Suppose $m$ is square-free and let $\beta \in \AA \cap K$. From the application of Theorem 13 (p. 28), we have an integral basis of $\AA \cap K$. This gives us two cases:

Case 1: $m\not\equiv\pm1\pmod{9}$. Then $\beta = a+b\alpha+c\alpha^2$ for some $a,b,c\in \ZZ$. So by (a) we have $\NrKQ(\beta) \equiv a^3 \pmod{m}$.

Case 2: $m\equiv\pm1\pmod{9}$. Then
$$
\beta = a + b\alpha + c \cdot \frac{\alpha^2 \pm \alpha + 1}{3}
= \left(a+\frac{c}{3}\right) + \left(b\pm\frac{c}{3}\right)\alpha + \left(\frac{c}{3}\right)\alpha^2
$$
for some $a,b,c\in \ZZ$. By (a) we have
$$
\NrKQ(\beta) \equiv a^3 + a^2c + \frac{ac^2}{3}(1 \mp m) + \frac{c^3}{27}(1\mp m)^2  \pmod{m}.
$$
For the case $m\equiv 1\pmod{9}$, we write $m = 9k+1$ for some $k\in\ZZ$. Then from the above, we have
\begin{align*}
\NrKQ(\beta) 
&\equiv a^3 + a^2c - 3kac^2 + 3k^2c^3  \pmod{9k+1}    \\
&\equiv a^3 - 9ka^2c + 27k^2ac^2 - 27k^3c^3  \pmod{9k+1}    \\
&= (a-3kc)^3 \pmod{m}.
\end{align*}
(Note that $-1 \equiv 9k \pmod{9k+1}$.) So $\NrKQ(\beta)$ is a cube mod $m$. A similar argument shows that when $m\equiv -1\pmod{9}$, $m = 9k-1$ for some $k\in\ZZ$, then $\NrKQ(\beta) \equiv (a+3kc)^3 \pmod{m}$.

\subsection*{Exercise 5.24}

Note that $R := \AA \cap \QQ(\sqrt[3]{7}) = \ZZ[\sqrt[3]{7}]$ and $\disc(R) = -27 \cdot 7^2$. So $\|J\| \leq 10$. (See Exercise 5.22 for more details.) Consider the primes lying over $2,3,5,7$ (use Exercise 3.26):
\begin{alignat*}{2}
2R &= P_2Q_2 \text{ with } f(P_2/2) = 1, f(Q_2/2) = 2 &&\implies \|P_2\| = 2, \|Q_2\| = 2^2,    \\
3R &= P_3^3 &&\implies \|P_3\| = 3,    \\
5R &= P_5Q_5 \text{ with } f(P_5/5) = 1, f(Q_5/5) = 2 &&\implies \|P_5\| = 5, \|Q_5\| = 5^2,    \\
7R &= P_7^3 &&\implies \|P_7\| = 7.
\end{alignat*}
Since $\|J\| \leq 10$, we shall only consider $P_2,Q_2,P_3,P_5,P_7$. By Exercise 5.21 we know the formula of norm of some particular elements in $R$:

Since $\|(\sqrt[3]{7})\| = |\Nr(\sqrt[3]{7})| = 7$, so $(\sqrt[3]{7}) = P_7$ is principal. We next claim that the remaining primes are not principal. For example, if $P_2 = (\beta)$ for some $\beta \in R$, then $|\Nr(\beta)| = \|P_2\| = 2$. So we have $\Nr(\beta) = \pm2 \equiv 2,5 \pmod{7}$. But this contradicts to Exercise 5.23(b) because $2,5$ are not cubes mod $7$. A similar argument works for the other primes. So the ideal class group has at least two elements. Moreover, this also implies that $\ovl{P_3}$ has order three, so the ideal class group has at least three ideal elements.

Since $\|(\sqrt[3]{7}-1)\| = |\Nr(\sqrt[3]{7}-1)| = 6 = 2\cdot 3$, we have $(\sqrt[3]{7}-1) = P_2P_3$. And since $\|(\sqrt[3]{7}+2)\| = |\Nr(\sqrt[3]{7}+2)| = 15 = 3\cdot5$, we have $(\sqrt[3]{7}+2) = P_3P_5$. Combining these two with the above factorizations, we see that $\ovl{P_2} = \ovl{P_3}^{-1} = \ovl{P_5}$ and $\ovl{Q_2} = \ovl{P_2}^{-1} = \ovl{P_3}$. Hence the ideal class group of $\ZZ[\sqrt[3]{7}]$ has three elements.

\subsection*{Exercise 5.25}

\subsection*{Exercise 5.26}

\subsection*{Exercise 5.27}

\subsection*{Exercise 5.28}

(a) By the Corollary 2 of Theorem 35 (p. 92), the ideal class group of $R$ is a finite group. So $I^m = \alpha R$ for some $m\in\NN$ and $\alpha \in R$. Set $L:=K(\sqrt[m]{\alpha})$. Then $L/K$ is a finite extension and $\sqrt[m]{\alpha} \in S$ as $\sqrt[m]{\alpha}$ is integral over $R$. Now, note that $(IS)^m = I^mS = \alpha S = (\sqrt[m]{\alpha}S)^m$. So $IS = \sqrt[m]{\alpha}S$ is principal by the unique factorization of ideals.

(b) Let $\{I_1,\ldots,I_h\}$ be a set of representatives from each ideal class of $R$ where $h$ is its class number. By (a), there exists a finite extension $L_i/K$ s.t. $I_i$ becomes principal in $L_i$. Let $L := L_1\cdots L_h$ be the compositum of $L_1,\ldots,L_h$. Then $L/K$ is a finite extension and each $I_i$ becomes principal in $L$

In general, for any ideal $I$ of $K$, we have $\alpha I = \beta I_i$ for some non-zero $\alpha,\beta \in R$ and $i=1,\ldots,h$. So $\ovl{(\alpha)} \cdot \ovl{IS} = \ovl{(\beta)} \cdot \ovl{I_i S} = \ovl{(\beta)}$ as $I_iS$ is principal. Hence, $IS$ is principal.

(c) Let $K:=\QQ(\sqrt{-21})$. From Exercise 5.14, we know the ideal class group of $\ZZ[\sqrt{-21}]$ is the Klein four group $\{\ovl{R},\ovl{P_2},\ovl{P_3},\ovl{P_5}\}$ where $P_2^2=2R$ and $P_3^2=3R$. So by the construction in (b), we see that $P_2,P_3$ become principal in $L:=K(\sqrt{2},\sqrt{3})$ where $[L:K]=4$. And since $\ovl{P_2}\cdot\ovl{P_3} = \ovl{P_5}$, so $P_5$ is also principal in $L$.

\subsection*{Exercise 5.29}

\begin{comment}
(a) By induction, it's sufficient to consider the ideal $I=(\alpha,\beta)$ in $\AA$ which is generated by two elements $\alpha,\beta \in \AA$. Take a field $K/\QQ$ s.t. $\alpha,\beta \in \AA\cap K=:R$ and consider the ideal $I':=(\alpha,\beta)$ in $R$ generated by $\alpha,\beta$. By Exercise 5.28 (b), choose $L/K$ finite s.t. $I'$ becomes principal in $S:=\AA \cap L$, i.e., $I'S = \gamma S$ for some $\gamma \in S$. This implies $I = I'S \AA = \gamma S \AA = \gamma \AA$ is principal in $\AA$.
\end{comment}

\subsection*{Exercise 5.30}

\subsection*{Exercise 5.31}

\subsection*{Exercise 5.32}

\subsection*{Exercise 5.33}

\subsection*{Exercise 5.34}

\subsection*{Exercise 5.35}

(a) Since $u$ is a unit, so $1 = |\Nr(u)| = |u \cdot \rho e^{i\theta} \cdot \rho e^{-i\theta}| = u\rho^2$ and hence $u=\rho^{-2}$.

We omit the tedious calculation. One may use Theorem 8 (p. 19) and the facts that $\sin\theta = (e^{i\theta}-e^{-i\theta})/(2i)$ and $\sin2\theta = 2\sin\theta\cos\theta$.

(b) Set $x=\rho^3+\rho^{-3}$ and $c=\cos\theta$. From (a) we know $u=\rho^{-2}$, and this implies that $u^3+u^{-3} = (\rho^3+\rho^{-3})^2-2 = x^2-2$. Now, when $c=0$ (so that $\theta=\pm\pi/2$), then we have $|\disc(u)| = 4(\rho^3+\rho^{-3})^2 = 4(u^3+u^{-3}+2)$. So the result is clear. Assume $c\neq0$. Then we see that the quadratic equation $f(x)=(1-c^2)(x-2c)^2-x^2$ has maximum $4(1-c^4)$. In others words, we have
$$
(1-c^2)(x-2c)^2-x^2 \leq 4(1-c^4) < 4.
$$
So
$$
|\disc(u)| = 4(1-c^2)(x-2c)^2 < 4(x^2+4) = 4(u^3+u^{-3} +6).
$$

(c) From (b) and Exercise 2.27(c), we have
$$
d := |\disc(R)| \leq |\disc(u)| < 4(u^3+u^{-3}+6).
$$
This gives the first inequality. For the second one, it's equivalent to the inequality $1<u^3$, which follows from the hypothesis that $u>1$.

(d) Suppose now $d\geq33$. By (c) we have $33 \leq d < 4(u^3+u^{-3}+6)$. So $u^3+u^{-3} > 9/4$. Solving the equality $u^3+u^{-3} = 9/4$ for $u^3$, it follows that $u^3 = (9+\sqrt{17})/8 > 13/8 > 4/3$. And this is equivalent to $-6-u^{-3} > -27/4$. So by (c),
$$
u^3 > \frac{d}{4} -6 - u^{-3} > \frac{d-27}{4}.
$$

\subsection*{Exercise 5.36}

Let $K:=\QQ(\alpha)$ and $R:=\AA \cap K = \ZZ[\alpha]$.

(a) $d := |\disc(R)| = |\disc(\alpha)| = 108 \geq 33$. So by Exercise 5.35(d), we have $u^3 > (108-27)/4 > 20$.

(b) From Exercise 5.23(a) we find that $\NrKQ(-1+\alpha) = 1$, so $\beta$ is a unit in $R$. Since by Theorem 38 (p. 100) we know $U = \{\pm u^k \mid k \in\ZZ\}$, so $\beta = u^k$ for some $k\in\NN$. The result $u^3 > 20$ in (a) implies that $u^2 > 20^{2/3}$. So we have $u^0 = 1 < \beta < 20^{2/3} < u^2$. This shows that $k=1$ and hence $\beta=u$.

\subsection*{Exercise 5.37}

(a) Note that the constant term of $g(x) := f(x+r)$ is $f(r) = \pm 1$ and $\alpha-r$ is a root of $g(x)$. So the norm of $\alpha-r$ is necessarily $\pm1$, which implies that $\alpha-r$ is a unit in $\AA$.

(b) By Exercise 2.41 (or just see the application of Theorem 13 (p. 28)), we know $R:= \AA\cap\QQ(\alpha) = \ZZ[\alpha]$. Note $\alpha$ is a root of the monic polynomial $f(x)=x^3-7$ and $f(2)=1$. So by (a) we know $\alpha-2$ is a unit in $R$.

By Exercise 2.41(a) we have $d := |\disc(R)| = |\disc(\alpha)| = 1323 \geq 33$. So by Exercise 5.35(d), $u^3 > (1323-27)/4 = 324$. Therefore, $u^2 > 324^{2/3} > 47$. Let $\beta:=(2-\alpha)^{-1}$, which is a unit in $R$. And note that $u^0 = 1 < \beta < 47 < u^2$, so $\beta=u$ is the fundamental unit in $R$.

(c) Similarly to (b), we see that $R:= \AA\cap\QQ(\alpha) = \ZZ[\alpha]$. Note $\alpha^2$ is a root of the monic polynomial $f(x)=x^3-9$ and $f(2)=-1$. So by (a) we know $\alpha^2-2$ is a unit in $R$.

By Exercise 2.41(a) we have $d := |\disc(R)| = |\disc(\alpha)| = 243 \geq 33$. So by Exercise 5.35(d), $u^3 > (243-27)/4 = 54$. Therefore, $u^2 > 54^{2/3} > 14$. Let $\beta:=(\alpha^2-2)^{-1}$, which is a unit in $R$. And note that $u^0 = 1 < \beta < 14 < u^2$, so $\beta=u$ is the fundamental unit in $R$.

\subsection*{Exercise 5.38}

(a) Let $f(x):=x^3+x-3$. Note that its discriminant $-4(1)^3-27(-3)^2 < 0$. So $f$ has only one real root $\alpha$. (This guarantees the desired structure of the unit group in (b) and (c) because there's only one real embedding.)

A simple calculation shows that $f(1.2) < 0 $ and $f(1.3) > 0$. So by the intermediate value theorem, $\alpha$ is between $1.2$ and $1.3$.

(b) Clearly $f$ is irreducible over $\QQ$. By Exercise 2.28(c), $\disc(\alpha) = -(4(1)^3+27(-3)^2) = -247 = -13\times 19$ which is square-free. So by Exercise 2.27(e), $\{1,\alpha,\alpha^2\}$ is an integral basis for $R$, which implies that $\disc(R) = \disc(\ZZ[\alpha]) = \disc(\alpha)$.

(c) Note $\alpha$ is a root of the monic polynomial $f(x)=x^3+x-3$ and $f(1)=-1$. So by Exercise 5.37(a) we know $\alpha-1$ is a unit in $R$.

By (b) we have $d := |\disc(R)| = |\disc(\alpha)| = 247 \geq 33$. So by Exercise 5.35(d), $u^3 > (247-27)/4 = 55$. Therefore, $u^2 > 55^{2/3} > 14$. Let $\beta:=(\alpha-1)^{-1}$, which is a unit in $R$. And note that by (a) we have
$$
u^0 = 1 < 0.3^{-1} < \beta < 0.2^{-1} < 14 < u^2.
$$
So $\beta=u$ is the fundamental unit in $R$.

\subsection*{Exercise 5.39}

Let $f(x):=x^3-2x-3$. Note that its discriminant $-4(-2)^3-27(-3)^2 < 0$. So $f$ has only one real root $\alpha$. And a simple calculation shows that $f(1.8) < 0 $ and $f(1.9) > 0$. So by the intermediate value theorem, $\alpha$ is between $1.8$ and $1.9$.

Clearly $f$ is irreducible over $\QQ$. By Exercise 2.28(c), $\disc(\alpha) = -(4(-2)^3+27(-3)^2) = -211$ which is square-free (in fact, a prime). So by Exercise 2.27(e), $\{1,\alpha,\alpha^2\}$ is an integral basis for $R$, which implies that $\disc(R) = \disc(\ZZ[\alpha]) = \disc(\alpha)$.

Note $\alpha$ is a root of the monic polynomial $f(x)=x^3-2x-3$ and $f(2)=1$. So by Exercise 5.37(a) we know $\alpha-2$ is a unit in $R$.

By (b) we have $d := |\disc(R)| = |\disc(\alpha)| = 211 \geq 33$. So by Exercise 5.35(d), $u^3 > (211-27)/4 = 46$. Therefore, $u^2 > 46^{2/3} > 12$. Let $\beta:=(2-\alpha)^{-1}$, which is a unit in $R$. And note that by (a) we have
$$
u^0 = 1 < 0.2^{-1} < \beta < 0.1^{-1} < 12 < u^2.
$$
So $\beta=u$ is the fundamental unit in $R$.

\subsection*{Exercise 5.40}

\begin{comment}
(a) Let $f(x):=x^3+ax-1$. The only possible rational roots of $f$ are $\pm1$, which clearly do not satisfy $f$. So $f$ is irreducible over $\QQ$. Note that its discriminant $-4a^3-27(-1)^2 = -4a^3-27 < 0$ for all $a\in\NN$. So $f$ has only one real root $\alpha$.

(b) By Exercise 2.28(c), $\disc(\alpha) = -(4a^3+27)$.

(c) Suppose $\disc(\alpha) = -(4a^3+27)$ is square-free. Then by Exercise 2.27(e), $\{1,\alpha,\alpha^2\}$ is an integral basis for $R$, which implies that $\disc(R) = \disc(\ZZ[\alpha]) = \disc(\alpha)$. Now, by (b) we have $d := |\disc(R)| = |\disc(\alpha)| = 4a^3+27 \geq 33$ for all $a \geq 2$. (Since $u>1$ so we can only consider $a\geq2$.) So by Exercise 5.35(d), $u^3 > (d-27)/4 = a^3$. Therefore, $u>a$. 

On the other hand, note $\alpha$ is a root of the monic polynomial $f(x)=x^3+ax-1$ and $f(0)=-1$. So by Exercise 5.37(a) we know $\alpha$ is a unit in $R$. Moreover, we see that $f(0)<0$ and $f(1)=a>0$. So by the intermediate value theorem, $\alpha$ is between $0$ and $1$.

Since $\alpha^{-1}>1$ is a unit, so $\alpha^{-1} = u^k > a$ for some $k\in\NN$. Moreover, the two facts $\alpha > \alpha^3$ and $\alpha^3+a\alpha-1=0$ give us $\alpha^{-1} < a+1$. So we have $\alpha^{-1}$ is between $a$ and $a+1$.

Suppose now $a\geq2$, then we have
$$
u^0 = 1 < a < \alpha^{-1} = u^k < a+1 < u+1 < u^2
$$
(As $u>a\geq2$, it's easy to check that $u+1<u^2$.) So $\alpha^{-1}=u$ is the fundamental unit in $R$.
\end{comment}

\subsection*{Exercise 5.41}

\subsection*{Exercise 5.42}

\subsection*{Exercise 5.43}

\subsection*{Exercise 5.44}

\subsection*{Exercise 5.45}

\subsection*{Exercise 5.46}

\subsection*{Exercise 5.47}

\subsection*{Exercise 5.48}
\phantom{}

\end{document}